\chapter{Moderni veb}

Internet je odavno postao sastavni deo života ljudi, a prvenstveno mladima.
Predstavlja neiscrpan izvor zabave i informacija i povezuje ljude širom sveta.
Ujedinjene nacije su 2016. godine proglasile pristup internetu osnovnim ljudskim pravom.
Gugl je još 2011. godine pustio u prodaju Hrombuk, lap-top u kome su operativni sistem i brauzer spojeni u jedno.
Sve više aplikacija se izvršavaju direktno u brauzeru, umesto klasičnih desktop aplikacija.

Ovo ima brojne prednosti, ali se sledeće dve izdvajaju kao ključne.

Veb-aplikacije od korisnika \textbf{ne zahtevaju instalaciju}.
Mnogo je veća verovatnoća da će potencijalni klijent koristiti aplikaciju ukoliko može da je koristi čim naide na nju, umesto da mora da je preuzme i instalira na svojoj mašini.
Sem toga, danas nije retkost da korisnici poseduju više uredaja koje svakodnevno koriste.
Veb-aplikacija je po definiciji dostupna sa svakog od njih, dok bi desktop- ili mobilnu aplikaciju trebalo instalirati na svakom uredaju ponaosob.

Pored ovoga, \textbf{ažuriranje} klasičnih desktop aplikacija zahteva neki vid akcije od korisnika.
Čak i ako je ažuriranje dobro isplanirano i može da se odvija automatski i bez čitave ponovne instalacije programa, korisnici uvek imaju mogućnost da ipak ostanu na staroj verziji.
Developer ne može da bude siguran u to koju verziju aplikacije koriste klijenti, što za posledicu ima razne probleme u kompatibilnosti prilikom, na primer, komunikacije sa serverom, koji bi morao da uvek podržava sve verzije desktop aplikacije.
Veb-aplikaciju je trivijalno ažurirati; garantovano je da će svi korisnici istog trenutka moći da pristupe jedino najnovijoj verziji.

\subfile{ch01/01}

\subfile{ch01/02}
