\section{Pomoćni alati i novi jezici}

Osim okvir\=a, razvijaju se i alati koji pomažu autorima prilikom pisanja koda.

\subsection{Moduli u Javaskriptu}
\label{subsec:js-moduli}

Kao glavni nedostatak Javaskripta se često navodilo to što nema specifikaciju za podelu koda u manje datoteke \cite[\S10]{eloquent-js}.
Najraniji alati koji su služili da nadomeste ovaj nedostatak su radili jednostavnu konkatenaciju \cite{ancienct-concat}.
Sa pojavom Noda se javio prvi formalni standard; koristi se funkcija \code{require} za uvoz datoteka i posebna globalna promenljiva \code{exports} za izvoz simbola iz njih.
Ovaj standard je kasnije dobio ime \textbf{CJS} (\textsl{CommonJS}).
Iako je ovaj standard bio specifičan za Nod i iako nije bio prihvaćen kao Ekmaskript standard, javili su se alati koji dopuštaju developerima da kod namenjen izvršenju u brauzerima (a ne u Nodu) pišu u odvojenim fajlovima i da reference na simbole definišu eksplicitno koristeći \code{require} i \code{exports} umesto da se oslanjaju na primitivne metode kao što je konkatenacija.
Ovi alati su bili prvi \textbf{bandleri}.
Kao izlaz su davali isključivo jednu \code{.js} datoteku, tako što su zavisnosti među izvornim datotekama praćene od jedne početne.

Aplikacije su vremenom rasle a SPA metoda za izgradnju aplikacija se koristila sve češće.
Postalo je jasno da se dobar deo koda koji se isporučuje korisnicima nikada ne izvrši, jer korisnik nikada ne dođe do određenog dela aplikacije.
Da bi se ovaj problem premostio, zajednica je definisala specifikaciju poznatu kao \textbf{AMD} (\textsl{Asynchronous Module Definition}).
Ovime je omogućeno da se definišu zavisnosti između datoteka u fazi izvršenja programa na klijentu i da se pribave asihrono, kada se za to javi potreba.
Pošto su zavinosti eksplicitno definisane, dodatne zavisnosti se pribavljaju automatski, pa se s developera skida teret da mora ručno da definiše koje datoteke se prve moraju učitati da bi se učitala druga.

Osim pisanja Javaskripta kroz AMD module, moguće je i pisati ih u odvojenim datotekama a prepustiti nekom devop alatu da ih pretvori u AMD module, bilo da su oni u jednoj datoteci ili u više njih.

I CJS i AMD su dosegli dovoljnu popularnost da se smatraju \textquote{standardom} -- jedan za Node, a drugi za brauzere.
Ali šta je sa modulima za koje ima smisla da istovremeno budu dostupni i u Nodu i u brauzeru; na primer, što su biblioteke sa pomoćnim funkcijama kao što je \textsl{lodash}?

Kako ne bi postojale dve različite verzije datoteka, osmišljena je nova specifikacija, nazvana \textbf{UMD} (\textsl{Universal Module Definition}).
UMD ne samo da je kompatibilan i sa CJS-om i sa AMD-om, već radi i sa globalno definisanim simbolima.

Paralelno sa ovim, sa porastom popularnosti Javaskripta i veba kao vodeće platforme za razviće aplikacija, počinje da se intenzivnije radi se na specifikaciji Harmonije, nove verzije EkmaSkript specifikacije koja je danas poznata kao ES2015.
Ovime je problem rada sa više fajlova formalno rešen.

Međutim, postoje dva razloga zašto većina developera ne koristi ES2015 module u produkciji.
Prvi je to što i dalje postoji određeni procenat ljudi koji koristi brauzere koji nisu implementirali ovaj standard.
Drugi je to što se povećava broj zahteva i aplikacija se usporava.

Osim toga što se zahteva više datoteka (što predstavlja više zahteva ka serveru pa samim tim dužu komunikaciju sa istim), ostaje problem minifikacije i drugih tehnika optimizacije koje ili nisu izvodljive ili nisu u dovoljnoj meri izražene.
Zbog ovoga se i dalje koriste bandleri koji na složeni način obarđuju fajlove kako bi pronašli njihove međuzavisnosti i od njih stvorile jedan ili više izlaznih datoteka spremnih za produkciju.

\subsection{Bandleri i minifikacija}

Banlovanje u Javaskriptu je tehnika koja grupiše odvojene datoteke u cilju smanjenja broja HTTP zahteva koje je potrebno otpočeti kako bi se stranica učitala.

Preteča bandlera su \textquote{izvršioci zadataka} (\textsl{task runners}), među kojima se ističu \textsl{gulp} i \textsl{grunt}.
Rešenje koje oni nude je daleko od idealnog -- developer mora da u nekoj vrsti konfiguracionog fajla definiše redosled kojim fajlovi treba da se učitaju, a izvršilac jednostavno konkatenira sve datoteke, kao da su proizvoljni stringovi, u jedan fajl koji je developer uključivao u stranicu.

Kao jedno od rešenja za ovaj problem nastaje \textbf{Vebpek} (\textsl{webpack}).
Kada procesira aplikaciju, on je potpuno svestan Javaskripta kao jezika u kome je napisana.
Kod se parsira, a u nekim slučajevima se delovi koda i interpretiraju; na osnovu toga se rekurzivno izgraduje graf zavisnosti koji sadrži svaki modul koji je potreban aplikaciji.
Kao krajnji proizvod daje manji broj paketa (nekad i samo jedan) koje će brauzer učitati.
Veoma je konfigurabilan, što omogućuje granularnije podešavanje veličine i
izgleda bandlova, kao i lenjo učitavanje tek kada ih korisnička aplikacija zahteva, čime se ubrzava inicijalno učitavanje aplikacije.

Sličan alat je \textbf{Rolap} (\textsl{Rollup}), mada je nastao iz drugačijih razloga.
Naime, Rolap je prvi bandler koji je prihvatio ES2015 module kao prvoklasni način za učitavanje modula, omogućivši pritom tzv. \textit{tree-shaking}~\cite{rollup-guide}.
U pitanju je proces kojim se iz modula dovlače samo simboli koji se koriste, umesto celog modula.
Ovo znatno smanjuje veličinu koda prilikom korišćenja eksternih modula, pogotovo modula koji pružaju veliki broj pomoćnih funkcija (kao što je \textit{lodash}).

Zbog koncepta klijenta na vebu, Javaskript ima specifične zahteve kada su performanse u pitanju.
Kod desktop aplikacija, pažnja se retko obraća na veličinu izvršne datoteke jer nije od važnosti -- korisnici su svesni da mora da prođe određeno vreme kada prvi put preuzimaju aplikaciju i kada je instaliraju, i razlika između sto i dvesta megabajta prolazi nezapaženo.
Performanse desktop aplikacija mere su vremenu potrebnom za izvršenje zadatka za koje su namenjene ili u brzini odziva tokom navigacije kroz korisnički interfejs.

Međutim, na vebu su stvari drugačije.
Pored istih mera performansi koje imaju desktop aplikacije, ključni faktor za performanse koje će korisnik doživeti je i veličina koda.
Da stvar bude komplikovanija, Javaskript je jezik koji se interpretira, što znači da se izvorni kod šalje direktno klijentu.
Ukoliko bi autor koda ručno probao da ga učini optimalnim za transfer preko mreže, posle svega nekoliko minuta bi bio neodrživ -- bilo bi potrebno ukloniti sve nepotrebne znake beline, što znači da bi se kod pisao u jednoj liniji, a sva imena promenljivih imala bi jedno ili dva slova.

Kako bi se rešio ovaj problem, razvijeni su \textbf{minifikatori}.
Tehnički, u pitanju su kompilatori koda, ali prevode Javaskript nazad \emph{u Javaskript}.
Autor može da tokom razvića aplikacije koristi neminifikovanu verziju, što pomaže prilikom debagiranja, a da pred slanja na produkciju pokrene minifikator.
Ipak, ovo je mukotrpan posao pa se minifikatori ugrađuju u bild-proces, obično u vidu plagina za bandlere kao što su Vebpek i Rolap.

Najpopularniji minifikator za Ekmaskript je \textbf{Terser} (\textsl{terser}).
Sastoji se od parsera, generatora koda, optimizatora, menglera, analizatora opsega, šetača stabla i transformera stabla \cite{uglify}.
Parser služi za kreirnje apstraktnog sintaksnog stabla (AST) na osnovu Javaskript koda, koji se dalje koristi od strane optimizatora, što je primarni transformer stabla.
Njime se obilazi dobijeni AST i vrše se transformacije nad njima na osnovu unapred utvrđenih pravila, sa ciljem sa se veličina stabla smanji.
Na primer: čvorovi koji predstavljaju konstantne izrazi kao što je \code{1 + 2} mogu biti sračunati od strane optimizatora i zamenjeni literalom koji predstavlja dobijenu vrednosti; ukoliko se u uslovu naredbe \code{if} nađe konstanta \code{true}, cela negativna grana i sama naredba \code{if} se mogu ukloniti.

Neke optimizacije nisu \textquote{sigurne}, odnosno mogu potencijalno da \textquote{slome} kod ukoliko se obave.
Jedna takva je promena imena promenljivih (ređe) i imena ključeva u svojstvima objekta (češće), koja se naziva menglovanje.
Ukoliko se ovi uslovi ispune, prolazak menglera kroz kod može drastično da smanji veličinu koda pukim prepisivanjem imena.

\subsection{Javaskript kao asembler veba}

Jezici koji se kompajliraju u Javaskript postaju popularni zajedno sa pojavom SPA kao obrasca za izradu aplikacija.
Prvi široko rasprostranjen takav jezik je \textbf{Kofiskript} (\textsl{CoffeeScript}) \cite{coffeescript-homepage}.
Nastao je 2009, a najveću popularnost ima od 2011 do 2014, kada polako počinje da pada u zaborav.
U SOJS anketi iz 2016, 60\% ispitanika je reklo da je \textquote{čulo za njega, ali ne želi da ga uči}, dok 25\% kaže da su \textquote{ga koristili ranije, ali više ne} -- samo 6\% je bilo zadovoljno jezikom.

Pad Kofiskripta može da se protumači kao pozitivna stvar, kada se uzme u obzir vreme i razlog zbog koga je on nastao.
Naime, razlog zbog koga je postao popularan bile su one strukture koje su kasnije došle uz Ekmaskript 2015.
Razlika je u tome što Kofiskript nije \textit{standard}, niti se oslanja na njega -- radi se o potpuno proizvoljnoj sintaksi, inspirisanoj Rubijem, Pajtonom i Haskelom.
Uz dolazak novih standarda, novih alata i novih jezika, Kofiskript nema šta da pruži.
Bez načina za definisanje tipova podataka, nema prednost u ranijem otklanjanju grešaka iz koda, a bez novih osobina u odnosu na novi ES standard, nema razloga za korišćenje.
Konciznija i kraća sintaksa, umesto prednosti, počinje da bude dodatni korak koji developeri moraju da savladaju pre nego što mogu da počnu sa radom \cite{gh:coffeescript-future}.

Ipak, Kofiskript igra značajnu ulogu u razviću samog Javaskripta.
Daglas Krokford izjavio je 2012. da je Kofiskript \textquote{lep, elegantan i minimalan} i da bi \textquote{voleo da je Javaskript više nalik Kofiskriptu}, nazivajući ga \textquote{ljupkim malim jezikom} \cite{crockford:java-was-failure}.
Brendan Ajk bio je šampion predloga za \textquote{debele streličaste funkcije} u Javaskriptu (\code{=>}), a deo ideje za konciznu sintaksu je dobio upravo od Kofiskripta \cite{hn:brendan-eich-dart-coffee-es6}.

Još jedan popularan jezik koji se pojavljuje je \textbf{Dart}; u pitanju je Guglov proizvod iz 2011. godine.
Kasnije dobija zvaničnu Ekma specifikaciju i poseban komitet, TC52.
Slično Kofiskriptu, ima proizvoljnu sintaksu -- ovog puta, inspirisanu C-olikim jezicima.
Umesto kraćeg zapisa Javaskript konstrukcija, Dart se osvrće na prvoklasnu podršku za obrasce kao što su klase, miksini, apstraktne klase, itd.
Pored ovoga, uvodi i sistem tipova -- sistem koji naziva \textquote{sigurnim}: koristi \textquote{kombinaciju statičke provere tipova i provere tokom izvršenja kako bi se postarao da vrednost koja je dodeljena promenljivoj uvek odgovara statičkom tipu promenljive} \cite{dart}.

Istovremeno, Majkrosoft radi na projektu pod kodnim imenom \textquote{Strada}.
Među timom od oko 50 ljudi koji su radili na projektu, ističe se ime Anders Hajlzberg -- autor Turbo Paskala i Delfija, i glavni arhitekta u timu za C$\sharp$ u Majkrosoftu.
Jezik je prvog oktobra 2012. godine objavljen pod nazivom \textbf{TajpSkript}, u verziji 0.8.
Prva stabilna verzija (1.0) objavljena je 2014. godine.
