\chapter{Moderni veb}

\epigraph{
  Samo hoću da prikažem podatke na stranici, ne da izvedem Sab-Zirov \textsl{fatality} iz originalnog MK.
}{\textquote{Kakav je osećaj učiti Javaskript u 2016}, Hose Aginaga \cite{hose:how-js-feels-2016}}

Teško je utvrditi tačan trenutak kada je internet nastao, ali se vezuje za pojavu elektronskih računara pedesetih i prve konceptualne prototipe regionalnih računarskih mreža šezdesetih \cite{history:internet}.
Ipak, internet kao mreža kakvu danas poznajemo vuče korene iz 1989, kada je Tim Berners-Li dobio ideju da sistem hipertekstualnih linkova, koji omogućuju skakanje sa jednog dokumenta na drugi, \textquote{samo poveže sa TCP-jem i DNS-om i --- tada! --- \textsl{World Wide Web}} \cite{lee:answers-for-young-people}.

Zatim je 1990. napisao prvu specifikaciju HTML-a koja je objavljena na internetu 1992. pod nazivom \textquote{HTML tagovi}, gde je naveo i opisao osamnaest tagova. Među njima su neki i danas prepoznatljivi, sa identičnim semantičkim značenjem: \code{title}, \code{a}, \code{p}, \code{h1} -- \code{h6}, \code{address}, \code{dl}, \code{dt}, \code{dd}, \code{ul} i \code{li} \cite{lee:html-tags}.

Osim toga, razvio je prvi brauzer -- \textit{WorldWideWeb}, kasnije preimenovan u \textit{Nexus}.
Brauzer je istovremeno imao i ugrađen WYSIWYG editor \cite{lee:www-browser}.

Stvari se potom odvijale brzo -- brauzeri su se takmičili za popularnost i međusobno smenjivali; HTML je dobio zvaničnu specifikaciju, iz godine u godinu dobijavši nove standarde; pojavio se CSS kao jezik za definisanje stilova; Javaskript postaje prvi programski jezik koji se izvršava u brauzeru.

\subfile{ch01/01}

\subfile{ch01/02}

\subfile{ch01/03}

\subfile{ch01/04}

\subfile{ch01/05}
