\section{Klasifikcija sistema tipova}

Skup pravila na osnovu kojih se svojstvo poznato kao \textbf{tip podatka} dodeljuje zaličitim delovima kompjuterskog programa (promenljive, izrazi, funkcije, moduli, itd) naziva se \textbf{sistem topiva podataka}.
Služe da bi se formalizovali koncepti koje programeri koriste za algebarske vrednosti, strukture podataka i druge komponente programa, kao što su \textquote{logička vrednost}, \textquote{niz stringova} i \textquote{funkcija koja vraća broj}.

Glavna namera postojanja tipova podataka jeste da minimizuje šanse za pojavu bagova u programima i pruži mogućnost za bolje isksutvo tokom razvoja softvera, ali se može koristiti i kao sredstvo za sprovođenje optimizacije.
\textbf{Provera tipova} (\textsl{type checking}) je proces kojim se utvrđuje da li su ispoštovana pravila i ograničenja koja nameću tipovi podataka u nekom programskom jeziku.
Postoji nekoliko osobina na osnovu kojih se može izvršiti podela načina na koje se vrši provera tipova.

\subsection{Podele}

Prvo pitanje koje se nameće jeste \textit{kada} se dešava provera: da li tokom kompilacije ili tokom izvršenja programa.
Za jezike koji se interpretiraju (Javaskript, Pajton, Rubi) ne postoji faza kompilacije pa je jedino mesto gde se može vršiti provera tipova -- tokom izvršenja.
Za ovakve jezike -- u kojima se provera tipova sprovodi tokom izvršenja -- kaže se da imaju \textbf{dinamičku} proveru tipova.
Interpretatori za ovakve jezike obično svakom objektu dodeljuju oznaku tipa u kojoj se nalaze informacije o tipu.
Ukoliko se provera vrši tokom kompjaliranja, radi se o \textbf{statičkoj} proveri.
Ako program zadovolji uslove koje nameće analizator kod jezika sa statičkom proverom tipova, onda to daje garanciju da program tokom izvršenja neće da naiđe na određeni skup mogućih grešaka u vezi sa tipovima podataka.

Poznajući trenutak izvršenja provere tipova podataka, postavlja se pitanje \textit{kako} se ona sprovodi.
Kod jezika sa \textbf{jakim} tipovima podataka, nad kodom su dozvoljene samo one operacije koje imaju smisla u skladu sa semantikom jezika i one koje neće dovesti do gubitka informacija.
Za jezike koje ne nameću ova ograničenja kaže se da imaju \textbf{slabe} tipove podataka.
Ne postoji jasna granica između jakih i slabih tipova, pa se ovi termini koriste uglavnom za poređenje jezika (koji ima \textquote{jače} a koji \textquote{slabije} tipove) ili za određene osobine jezika (za koju se može reći da je \textquote{jaka} a za koju \textquote{slaba}).

\textbf{Javaskript} nema podršku za statičko definisanje tipova podataka, a uz to je i slabo tipiziran jezik.
Promenljiva kojoj je dodeljen string može u sledećem redu da ima referencu na niz brojeva, a zatim da u nju bude upisana logička promenljiva.
Kako je Javaskript jezik koji se interpretira, ni nema prilike da se ovo zabeleži kao greška prilikom kompajliranja; taj proces ne postoji.
Međutim, čak i tokom izvršenja programa, ovakvo ponašanje koda se neće tretirati kao greška i interpretator će bez problema upisivati bilo koju vrednost u promenljivu.

Ovakvo ponašanje ne samo da može negativno da utiče na performanse (jer optimizator nije u stanju da predvidi koliko memorije je potrebno da se zauzme za promenljivu), već čini kod nečitljivim i podložnim greškama.
Glavna prednost Tajpskripta je uvođenje tipova podataka u Javaskript, ali samo prilikom kompajliranja.

Tajpskript ni na koji način ne utiče na fazu interpretiranja i izvršenja koda.
Tajpskript dolazi uz kompajler koji postojeći Tajpskript kod kompajlira u Javaskript.
Dobijeni Javaskript kod ne sadrži nikakve informacije i tipovima podataka; ne postoje nikakvi uslovi koji će dovesti do greške ukoliko se promenljivoj dodeli drugačiji tip od navedenog.
Umesto toga, Tajpskript ima zadatak da tokom kompajliranja utvrdi tipove i da se postara da developera obavesti o nastaloj nekonzistentnosti tako emituje grešku.

Da bi se ovo postiglo, Tajpskriptov kompajler obavlja statičku analizu koda tokom koje zaključuje tipove i proverava da li su određeni operatori izvodljivi nad njima.
Na primer, iako je se Javaskript interpretator ne buni za pokušaj sabiranja praznog niza i praznog objekta (\code{[] + \{\}}), Tajpskript će prijaviti grešku \textquote{Operator \textquote{\code{+}} cannot be applied to types \textquote{\code{undefined[]}} and \textquote{\code{\{\}}}}.

\subsection{Deklaracije}

Kako bi bilo moguće da se koriste postojeće biblioteke pisane u Javaskriptu, u Tajpskript je uveden pojam \textit{deklaracije}.
Na primer, TajpSkipt ne može automatski da bude svestan postojanja eksterne biblioteke učitane kroz \code{script} tag u \code{head} sekciji HTML dokumenta za koju se piše Tajpskript kod.
Deklaracija promenljive ili nekog drugog simbola navodi se ili u posebnom fajlu sa ekstenzijom \code{.d.ts} ili direktno u postojeći fajl, uz prefiks \code{declare}.

Deklaracije za mnoge postojeće Javaskript biblioteke se mogu naći na GitHab repozitorijumu Majkrosoftovog projekta \textsl{DefinitelyTyped} koji održava zajednica.
Svi paketi se mogu preuzeti sa npm repozitorijuma.
Radi lakšeg pronalaska deklaracionih fajlova koristi se imenski prostor (\textsl{namespace}) \code{@types}, a za ime deklaracionog paketa se koristi isto ime koje je dodeljeno samom paketu za koji se instaliraju deklaracije.
Na primer, za paket \code{jquery} deklaracije se nalaze na \code{@types/jquery}.

Deklaracije se mogu dostaviti i zajedno sa samim paketom; ovaj način se preferira u situacijama kada je sam paket već napisan u Tajpskriptu.
Koristeći opciju \code{--declaration}, Tajpskriptov kompajer može, pored \code{.js} fajlova, generisati i \code{.d.ts} fajlove, a iz \code{package.json} vrednosti pod ključem \code{types} može se pročitati putanja do ovog fajla.
Ovako Tajpskriptov kompajler zna gde da pronađe potrebne deklaracije.

\subsection{Definisanje tipova}

Neki stariji jezici sa statičkim tipovima podataka od developera zahtevaju da se tipovi uvek definišu eksplicitno.
Tajpskript je specifičan po tome što su se njegovi stvaraoci postarani da održe barijeru prelaska od Javaskripta na TajkSkript izuzetno niskom.
To je postignuto skroz nekoliko osobina.

Uz odgovarajuća podešavanja kompajlera, migracija na Tajpskript se sastoji od samo jednog koraka: promeniti ekstenziju svim \code{.js} datotekama u \code{.ts}.
Ovo je omogućeno činjenicom da su tipovi podataka \textbf{opcioni}.
Svaki Javaskript kod može se formalno prevesti u Tajpskript kod promenom ekstenzije, a Tajpskriptov kompajer će u procesu kompajliranja od \code{.ts} datoteke generisati kod identičan polaznoj \code{.js} datoteci.
Tajpskript je zbog ovoga s namerom zamišljen kao \textbf{nadskup} Javaskripta.

Za \textbf{eksplicitno} definisanje tipova koristi se postfiksna sintaksa koja je popularna među jezicima koji pružaju opcione tipove (npr. \textsl{EkšnSkript} i F$\sharp$).
Ispred tipa se navode dve tačke (\code{:}).

\begin{verbatim}
const foo: string = "Hello"
\end{verbatim}

Informacije o tipovima često nije neophodno ručno specificirati jer tipovi u Tajpskriptu mogu biti \textbf{implicitni}.
Analizator će, gde god je to moguće, na osnovu analize koda zaključiti o kom je tipu reč.
Na primer, ukoliko se konstanta inicijalizuje brojnom vrednošću, za konstantu se implicitno zaključuje da ima tip \code{number}.
Slično, ako se iz funkcije koja prima dva stringa vrati vrednost dobijenja njihovom konkatenacijom pomoću polimorfnog operatora \code{+}, povratnoj vrednosti funkcije se implicitno dodeljuje tip `string` jer je upravo tog tipa rezultat primene ovog operatora nad dva stringa.

\begin{verbatim}
const foo = 2103
const concat = (a: string, b: string) => a + b
\end{verbatim}

\subsection{Strukturalni sistem tipova}

Tradicionalni objektno-orijentisani jezici kao što su Java i C koriste sistem tipova u kome se podudaranje između tipova podataka vrši na osnovu njihovih eksplicitno navedenih \textit{imena}.
Ovakvi sistemi nazivaju se \textbf{nominalni}.
Na primer, iako su definisane dve strukture istog oblika, njihova međusobna dodela neće biti moguća jer su imena tipova različita.

% c
\begin{verbatim}
struct Foo { int baz; };
struct Bar { int baz; };

int main () {
  Foo foo;
  Bar bar;
  foo = bar; // error: no viable overloaded '='
}
\end{verbatim}

S druge strane su jezici kod kojih je za određivanje privila u sistemu tipova presudna njihova \textit{struktura}.
Oni se nazivaju \textbf{strukturalni} sistemi, u koje spada Tajpskript.

\begin{verbatim}
interface Foo { baz: number }
interface Bar { baz: number }

let foo!: Foo
let bar!: Bar
foo = bar // ok
\end{verbatim}
