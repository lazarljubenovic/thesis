\section{Tajpskriptov kompilator}

Tajpskriptov kompajler se u izvornom kodu Tajpskripta \cite{gh:ms:ts} nalazi u direktorijumu \code{src/compiler}.
Ključni delovi kompajlera su skener, parser, bajnder, čeker i emiter \cite{typescript-deep-dive}.

Izvorni kod programa pisanog u Tajpskriptu najpre se procesira od strane \textbf{skenera} (\textsl{scanner}, čitač).
Izlaz faze skeniranja je niz skeniranih tokena.
Ovako dobijeni tokeni se zatim obrađuju pomoću \textbf{parsera} (\textsl{parser}, raščlanjivač) i kao izlaz se dobija apstraktno sintaksno stablo.
Da bi se iz AST-a dobili simboli, koristi se \textbf{bajnder} (\textsl{binder}, povezivač).
Simboli su osnovni gradivni blok semantike koda -- na osnovu njih se, na primer, na osnovu upotrebe simbola prepoznati njegova deklaracija.

Zatim se \textbf{čeker} (\textsl{checker}, proveravač) koristi kako bi se validirali tipovi podataka.
Čeker koristi AST i simbole.
\textbf{Emiter} (\textsl{emitter}, objavljivač) na osnovu AST-a na izlaz daje Javaskript datoteke.

\begin{center}
  \begin{tikzpicture}[node distance = 1.5cm, auto]
    % Čvorovi
    \node [fc:block]                         (izvorni-kod) {izvorni kod};
    \node [fc:block, right = of izvorni-kod] (niz-tokena)  {niz tokena};
    \node [fc:block, right = of niz-tokena]  (ast)         {AST};
    \node [fc:block, right = of ast]         (simboli)     {simboli};

    \node [fc:block, below = of simboli]     (valid) {validacija};
    \node [fc:block, below = of ast]         (js)    {\code{.js}};

    % Veze
    \path [fc:line] (izvorni-kod) -- node {skener} (niz-tokena);
    \path [fc:line] (niz-tokena) -- node {parser} (ast);
    \path [fc:line] (ast) -- node {bajnder} (simboli);

    \path [fc:line] (ast) -- (valid);
    \path [fc:line] (simboli) -- node [left] {čeker} (valid);
    \path [fc:line] (ast) -- node [left] {emiter} (js);
    \path [fc:line, dashed] (valid) -- (js);
  \end{tikzpicture}
\end{center}
