\section{\code{ts-simple-ast}}
\label{sec:ts-simple-ast}

\subsection{Sintaksne vrste}

U ovom odeljku iznesene su neke karakteristične sintaksne vrste koje se javljaju u AST-u Tajpskript koda i ključne su za razumevanje poglavlja \cref{ch:kompilacija-vejn-aplikacije}.

\subsubsection{\code{Identifier}}
\label{ast:Identifier}

Identifikatori se često pojavljuju u AST-u kao listovi.
Koriste se prilikom deklaracije i definicije, kao i prilikom pristupa.
Na primer, definicija funkcije oblika \code{function foo () {}} na posle ključne reči \code{function} (\code{FunctionKeyword}) ima identifikator (\code{Identifier}).

Jedna od najznačajnijih metoda kad instancom identifikatora je \code{getText}.
Ova metoda služi za dobavljanje teksta koji se koristi za identifikator, tj. njegovog imena.

U narednom primeru identifikatori su podvučeni.
\code{this} nije identifikator jer je predstavljen posebnim tokenom \code{ThisKeyword}.

\begin{lstlisting}[keywords={baz, window, global, foo, bar, val, toString},keywordstyle={\underbar}]
const baz = window.global
function foo (bar) {
  const val = bar + baz + this.toString()
  return val
}
\end{lstlisting}

\subsubsection{\code{PropertyAccessExpression} i \code{ElementAccessExpression}}
\label{ast:PropertyAccessExpression}
\label{ast:ElementAccessExpression}

U Javaskriptu postoje dva osnovna načina da se \textbf{pristupi vrednosti} definisanoj u objektu.
Pristup se uvek vrši na osnovu ključa, ali se to sintaksno može zapisati na dva načina.
U velikom broju slučajeva koristi se operator \code{.} (tačka, \code{DotToken}), ali postoji i alternativni način kojim se pokriva mnogo veći broj slučajeva korišćenja, a to je korišćenje para uglastih zagrada (\code{[} -- \code{OpenBracketToken}, \code{]} -- \code{CloseBracketToken}).

Korišćenje tačke predstavljeno je čvorom \code{PropertyAccessExpression}, a korišćenje zagrada -- čvorom \code{ElementAccessExpression}.

\code{PropertyAccessExpression} ima metode \code{getExpression} i \code{getName}.
\code{getName} se odnosi na desnu stranu i povratna vrednost ove metode je uvek \code{Identifier} (v.~\cref{ast:Identifier}).
Leva strana dobija se pozivom \code{getExpression} i može da bude predstavljena različitim tipovima, uključujući i \code{Identifier} -- na primer, u slučaju lanca poziva (), može biti ponovo \code{PropertyAccessExpression}, u opštem slučaju može biti bilo kakav izraz ukoliko se smesti u zagrade (\code{Parenthesized\-Expression}).

S druge strane, \code{ElementAccessExpression} i sa desne ima mnogo veći spektar potencijalnih čvorova, jer se u paru uglasnih zagrada može naći bilo kakav izraz (kome se pristupa metodom \code{getArgumentExpression}).

\subsubsection{\code{CallExpression}}
\label{sec:sk:call-expression}

Odnosi se na izraze koji predstavljaju \textbf{pozive} funkcija i metoda.
Prepoznaju se po prisustvu para oblih zagrada unutar kojih se može naći proizvoljan broj argumenata.

\code{ts-simple-ast} nudi metodu \code{getExperssion} kojom se jednostavno može pribaviti čvor izraza s leve strane zagrada.
Najjednostavniji primer je poziv funkcije definisan direktno na osnovu identifikatora i u ovom slučaju je rezultat poziva metode \code{getExpression} čvor tipa \code{Identifier}, nad kojim se može pozvati \code{getText()} za pribavljanje imena čvora.
\code{getExpression} može biti i \code{PropertyAccessExpression} kada se za pristup funkcije koristi operator \code{.} (tačka) i \code{ElementAccessExpression} kada se koristi \code{[]} (par uglih zagrada između kojih se navodi izraz čijom se evaluacijom dobija pristup funkciji).
Pored toga, izraz sa leve strane može biti i potpuno proizvoljan \code{Expression} -- na primer, to može biti novi \code{CallExpression} (za kari metodu izvršenja funkcija) ili pak čitava deklaracija funkcije, koja može biti proizvoljno složena.

\begin{verbatim}
foo()
this.foo(21)
this[symbol]()
foo()()()
(() => 21)()
\end{verbatim}

\subsubsection{Tokeni dodele}\label{sec:sk:tokeni-dodele}

Mada se dodela najčešće vrši korišćenjem znaka jednakosti (\code{EqualsToken}), postoji ukupno trinaest tokena kojima se dodela može obaviti.
U prvoj koloni sledeće tabele pobrojani su tokeni koji predstavljaju operatore koji se koriste u dodelama, a u drugoj koloni se nalazi po jedan ilustrativni primer za svaki token.

\begin{tabularx}{\textwidth}{@{}ll@{}}
  \toprule
  \textbf{Vrsta sintakse}                             & \textbf{Primer} \\
  \midrule
  \code{EqualsToken}                                  & \code{state = 21} \\
  \code{PlusEqualsToken}                              & \code{state += 21} \\
  \code{MinusEqualsToken}                             & \code{state -= 21} \\
  \code{AsteriskEqualsToken}                          & \code{state *= 21} \\
  \code{SlashEqualsToken}                             & \code{state /= 21} \\
  \code{PercentEqualsToken}                           & \code{state \%= 21} \\
  \code{AsteriskAsteriskEqualsToken}                  & \code{state **= 21} \\
  \code{LessThanLessThanEqualsToken}                  & \code{state <= 21} \\
  \code{GreaterThanGreaterThanEqualsToken}            & \code{state >= 21} \\
  \code{GreaterThanGreaterThanGreaterThanEqualsToken} & \code{state >{}>= 21} \\
  \code{AmpersandEqualsToken}                         & \code{state \&= 21} \\
  \code{CaretEqualsToken}                             & \code{state \^{}= 21} \\
  \code{BarEqualsToken}                               & \code{state |= 21} \\
  \bottomrule
\end{tabularx}

\subsection{Primer}

\begin{verbatim}
class C {
  private m () {
    this.m()
  }
}
\end{verbatim}

\begin{forest}
  for tree={
    font=\ttfamily\footnotesize,
    grow'=0,
    child anchor=west,
    parent anchor=south,
    anchor=west,
    calign=first,
    edge path={
      \noexpand\path [draw, \forestoption{edge}]
      (!u.south west) +(7.5pt,0) |- node[fill,inner sep=1.25pt] {} (.child anchor)\forestoption{edge label};
    },
    before typesetting nodes={
      if n=1
        {insert before={[,phantom]}}
        {}
    },
    fit=band,
    before computing xy={l=15pt},
  }
[SourceFile
  [SyntaxList
    [ClassDeclaration
      [ClassKeyword]
      [Identifier]
      [OpenBraceToken]
      [SyntaxList
        [MethodDeclaration
          [SyntaxList
            [PrivateKeyword]
          ]
          [Identifier]
          [OpenParanToken]
          [SyntaxList]
          [CloseParanToken]
          [Block
            [OpenBraceToken]
            [SyntaxList
              [ExpressionStatement
                [CallExpression
                  [PropertyAccessExpression
                    [ThisKeyword]
                    [DotToken]
                    [Identifier]
                  ]
                  [OpenParanToken]
                  [SyntaxList]
                  [CloseParanToken]
                ]
              ]
            ]
            [CloseBraceToken]
          ]
        ]
      ]
      [CloseBraceToken]
    ]
  ]
  [EndOfFileToken]
]
\end{forest}
