\chapter{Mogućnosti Vejna}

Vejn je UI frejmvork koji služi za pisanje jednostraničnih veb-aplikacija u Tajpskriptu.
Uz frejmvork dolazi CLI i bandler sa optimizatorom, pa se ne javlja potreba za instaliranjem drugih paketa i podešavanjem sistema za kreiranje produkcionog bilda.

\section{Šabloni}

% ////
% TODO OVO TREBA DA IDE U DEO GDE SE OBJAŠNJAVA KAKO VEJN RADI

% Ovaj zadatak sam po sebi ne predstavlja veliki izazov, ali stvari postaju znatno komplikovanije kada se uvrsti nekoliko drugih zahteva.

% - Koliko se često ažurira pogled?
% - Koliko je efikasno njegovo ažuriranje?
% - Šta sve developer treba da uradi kako bi se ažuriranje dogodilo?
% - Koji su preduslovi koje strukture podataka moraju da ispune da bi se promena detektovala?

% Dati odgovre na ova pitanja znači uporediti UI frejmvorke. 
% ////

S obzirom da se o frejmvorku za kreiranje \emph{veb}-aplikacija, način komunikacije sa korisnicima odvija se pomoću DOM-a.
Brauzer kreira DOM na osnovu HTML-a poslatog sa servera, ali i pruža programski pristup istom pomoću Javaskripta.
Ideja korišćenja UI frejmvorka je da se zaobiđe ovaj proces
Glavni zadatak Vejna je da pogled sa kojim korisnik interaguje održava sinhronizovanim sa modelom, tj. internim stanjem aplikacije.
Ovo je omogućeno definisanjem oblika interfejsa kroz \emph{deklarativnu} sintaksu -- u pitanju je sintaksa koja podseća na HTML kako bi se developer osećao \textquote{kod kuće}, ali se zapravo radi o proizvoljnoj sintaksi.

Termin \textquote{deklarativna} se odnosi na to da developer ne upravlja t\^okom izvršenja koda tokom pisanja šablona.
Upravljanje tokom je redosled u kome računar izvršava izraze u skripti \cite{mdn:glossary:control-flow}.
Umesto ovoga, developer samo \emph{deklariše} \textbf{šta} treba da bude prikazano na ekranu korisnika na osnovu stanja modela, a ne i \textbf{kako} se ovaj proces odvija.
Ovo ne samo da olakšava pisanje šablona, već i omogućuje da se vrše interne promene u frejmvorku a da developer toga ne bude svestan -- drugim rečima, developer može prostim ažuriranjem verzije frejmvorka koju koristi da unapredi performanse aplikacije.

Šablonska sintaksa je nadskup HTML-a\footnote{Šabloni predstavljaju isečke korisničkog interfejsa, tj. ne radi se o čitavom HTML dokumentu. Pod \textquote{podskupom HTML-a} se misli na podskup sintakse koja je dozvoljena tamo gde se očekuje tzv. \textsl{flow content}; v. sekciju 3.2.4.2.2. specifikacije HTML 5.2. U praksi ovi detalji ne predstavljaju prepreku u razumevanju sintakse šablona, pa se koristi termin \textquote{podskup HTML-a}.} -- to znači da svaki validan HTML predstavlja validan Vejn šablon.
Ovo za posledicu ima mogućnost da se postojeći statički sajt vrlo brzo može integrisati u aplikaciju koja se piše u Vejnu, kao i da šablone mogu pisati ljudi koji nemaju iskustva sa pisanjem programa (npr. dizajneri, urednici).

\subsection{Vezivna sintaksa}

Proširenje sintakse HTML-a se ugleda u mogućnosti da se trenutno stanje modela sinhroniše sa pogledom, tj. da se korisnički interfejs osveži svaki put kada dođe do promene modela, u skladu sa pravilima definisanim u šablonima.
Ovo je omogućeno vezivnom sintaksom.

Vezivnom sintaksom se mogu definisati izrazi i iskazi.
Izrazi su literali i reference na svojstva iz klase, a iskazi su pozivi metoda iz klase.

\textbf{Literal} se definiše pukim navođenjem.
Među literale spadaju stringovi (\code{"foo"}, \code{'foo'}), brojevi (\code{21}, \code{0x3}), logičke konstante (\code{true} i \code{false}), \code{null} i \code{undefined}.
Mogu se referencirati \textbf{svojstva} sa klase za koju je šablon vezan ili alijasi koje izlaže direktiva u određenom opsegu, kao i \textbf{lanci pristupa} (\code{item}, \code{item.name}).
Ispred svojstva i lanca pristupa se može dodati znak uzvika (\code{!}), što znači da će nad podatkom biti primenjen \textbf{operator negacije}.

Od iskaza se mogu koristiti jedino \textbf{pozivi}. Argumenti se navode u malim zagradama, odvojeni zarezom -- kao i u Javaskriptu.
Pored izraza, kao argument se može navesti i specijalan simbol \code{\#} koji označava preuzimanje reference koja se prosleđuje kroz događaj (iz HTML elementa) ili izlaz (iz komponente).

Proizvoljni Javaskript izrazi nisu dozovljeni u vezivnoj sintaksi -- operator negacije predstavlja jedini vid manipulacije koji je moguć nad podacima u okviru definicije šablona.
Na primer, za šablon se ne može vezati izraz kao \code{user \&\& user.name} ili \code{name.length > 0}.
Ukoliko developer želi da u šablonu veže kompleksni izraz koji se računa na osnovu postojećih podataka iz modela, treba da koristi getere na klasi.

% TODO: ovo ide u poređenju sa frejmvorcima
% Na ovaj način šabloni ostaju čisti i povlači se jasna granica: s jedne strane se nalazi aplikaciona i biznis \emph{logika}, a sa druge strane se nalazi deklarativna definicija \emph{pogleda}.
% jsx frejmvorci čak i nemaju deklarativne šablone, a oni koji imaju (vue i angular) imaju kompleksna pravila za to ŠTA je uopšte deklarativno (https://angular.io/guide/template-syntax#template-expressions) da bi zabranili sporedne efekte, ili pak ignorišu to što neko može da uradi to (https://jsfiddle.net/50wL7mdz/643256/)
% ...a onda se jave zvanična linter pravila koja mogu da ograniče kompleksnost izraza (https://github.com/mgechev/codelyzer/pull/512, https://github.com/mgechev/codelyzer/pull/509), gde se koriste neke proizvoljne i subjektivne mere kao što je "predugačko", i sl.
% osim toga, testiranje postaje komplikovano ako se previše stvari nađe u šablonu.
% da bi se sve ovo izbeglo, Vejn nema to.

\subsection{Interpolacija}

Osnovni vid vezivanja modela jeste ispisivanje stringa direktno na korisnički interfejs.
Ovaj vid komunikacije modela i pogleda naziva se interpolacija.
Vezivna sintaksa se piše između para dvostrukih vitičastih zagrada.

Na primer, na ekran se može ispisati korisničko ime prijavljenog korisnika.

\begin{verbatim}
Welcome, {{ username }}!
\end{verbatim}

\subsection{HTML elementi}

Kao što je već rečeno, Vejn šabloni su nadskup HTML-a, pa se HTML elementi u šablonima mogu koristiti na standardan način.
Ovo uključuje definiciju atributa, prazne elemente (\textsl{void elements}) i izostavljanje tagova.

\begin{verbatim}
<form>
  <label for="name">Name</label>
  <input type="text" id="name" name="name">
</form>
\end{verbatim}

Slično parserima HTML-a kod modernih brauzera, parser šablona u Vejnu je otporan na greške -- na primer, nezatvoreni tagovi (koji moraju biti zatvoreni po specifikaciji) će se automatski zatvoriti, ali nije definisano tačno u kom trenutku (nailazak na sledeći element, kraj šablona, itd), pa se na ovo ponašanje ni u Vejn šablonima ne treba oslanjati.

Navođenjem atributa u HTML-u se definiše koja će svojstva i koje atribute imati odgovarajući čvor u DOM-u.
Svojstva i atributi se koriste da finije odrede semantiku i ponašanje elemenata.
Na primer, svojstvo \code{type} na \code{<input>} elementu određuje kakav se unos očekuje od korisnika.

Svojstva se mogu sinhronizovati sa modelom tako što se ime svojstva navede u uglastim zagradama.
U tom slučaju se vrednost koja se navodi sa desne strane imena tumači kao izraz u vezivnoj sintaksi.

\begin{verbatim}
<p [id]="myParagraph.uniqueId">...</p>
\end{verbatim}

Mapiranje između atributa HTML-a i svojstava DOM čvorova nije uvek jedan-na-jedan.
Na primer, mada je \code{<label for="anId">} validna definicija atributa \code{for}, ova informacija se sa čvora čita (i postavlja) preko svojstva \code{htmlFor}.
Kako bi se napravila eksplicitna razlika između svojstava i atributa čvorova, za definisanje atributa koristi se prefiks \code{attr}.
Ovo je slično razlici između korišćenja direktnog pristupa svojstvu (\code{element.svojstvo}) i korišćenju metode \code{setAttribute} nad čvorom.

\begin{verbatim}
<label [htmlFor]="anId">...</label>
labelEl.htmlFor = anId

<label [attr.for]="anId">...</label>
labelEl.setAttribute('for', anId)
\end{verbatim}

U slučaju da ime atributa u šablonu sadrži crticu, ono se automatski tumači kao atribut čvora (a ne njegovo svojstvo) jer više ne postoji dvosmislenost -- ime svojstava HTML elemenata su izabrana tako da ne sadrže crtice.

\begin{verbatim}
<input [aria-label]="aLabel">
\end{verbatim}

Interpolacija, atributi i svojstva se koriste sa izrazima vezivne sintakse.
Koriste se za vezivanje \emph{podataka} za poglede i očuvanje sinhronizacije.
Da bi se delalo na korisnikovu interakciju sa interfejsom, potrebno je pretplatiti se na događaje koje emituju HTML elementi.
Za ovo se koriste iskazi vezivne sintakse; dakle, definišu se pozivi metoda
Drugima rečima, za šablon se vezuje \emph{ponašanje} aplikacije.

Za vezivanje metoda koje treba da budu pozvane na određeni događaj, koristi se sličan zapis kao definisanje svojstava i atributa.
Naime, ime događaja se omeđuje parom oblih zagrada, a s desne strane se navodi iskaz kojim se poziva metoda.

\begin{verbatim}
<button (click)="inc()">Increment</button>
\end{verbatim}

Za argumente funkcije se, osim iskaza, može navesti i poseban simbol \code{\#} -- \textsl{placeholder}.
Njime se može pristupiti objektu koji emituje događaj kako bi se informacije koje on nosi sa sobom mogle iskoristiti u telu metode.

\section{Komponente}

Kao i kod svih UI frejmvorka, glavna gradivna jedinica u Vejnu je \emph{komponenta}.
Komponente predstavljaju izolovane logičke celine koje se mogu smestiti u korisnički interfejs.

Komponentu treba shvatiti kao prirodnu nadogradnju HTML postojećih HTML elemenata.
Na primer, HTML element \code{HTMLInputElement} ima svoje \textbf{ime} na osnovu kojeg se prepoznaje u HTML kodu (string \code{input}), ima neka \textbf{svojstva} na osnovu kojih može da se kontrolišu finese njegovog ponašanja i izgleda koja se mogu specificirati navođenjem HTML atributa ili direktnom manipulacijom kad odgovarajućim propertijima Javaskript objekta (npr. \code{type}), i emituje neke \textbf{događaje} uz informaciju o tim događajima na koje se potrošači mogu pretplati i reagovati na njih.

Analogno tome, komponente dobijaju \textbf{ime} registrovanjem za druge komponente, \textbf{ulazi} (\textsl{inputs}) koji su sačinjeni od skupa svojstava klase koje je autor komponente proglasio kao delom njenog javnog API-ja, dok su \textbf{izlazi} događaji na koje se druge komponente mogu pretplatiti da bi se obavljala komunikacija između njih.

\subsection{Deklaracija}

Komponente se u Vejnu definišu klasom nad kojom je primenjen dekorator \code{@Template}, uvezen iz ulazne tačke modula \code{wane}.
Prvi i jedini argument dekoratora je string literal kojim se definiše struktura korisničkog interfejsa komponente.

Za stanje komponente koristi se telo klase, gde se standardno mogu definisati njena svojstva i metode.
Sva svojstva i metode su dostupne za referenciranje u šablonu.

\subsection{Pristupna komponenta}
\label{subsec:pristupna-komponenta}

Kao što se HTML-om definiše stablo elemenata, u Vejnu se od komponenti pravi \emph{stablo komponenti}.
U korenu takvog stabla leži posebna komponetna koja se naziva \textbf{pristupna komponenta} (\textsl{entry component}).

Pristupnu komponetu treba smestiti u datoteku \code{index.ts} u folderu \code{src} i načiniti je podrazumevanim izvozom iz modula -- ovako će je Vejn prepoznati kao ulaznu tačku u aplikaciju.
Kada se aplikacija pokrene, renderuje se ova komponenta, a ona za sobom povlači sve druge komponente kroz svoj šablon.

\subsection{Korišćenje}

Za razliku od HTML elemenata, koji su uvek globalno dostupni, komponente je potrebno \textbf{registrovati} za komponentu kako bi moglo da joj se pristupi iz šablona.
Registracija se vrši navođenjem reference na odgovarajuću klasu među argumentima dekoratora \code{@Register} koji se navodi uz klasu.
Ime simbola pod kojim se klasa registruje mora ima veliko početno slovo (tzv. \textsl{PascalCase}) kako bi se obezbedilo da neće doći do preklapanja sa imenima postojećih HTML elemenata i potencijalnih veb komponenti koje su registrovane na istoj stranici.

Osim razlike u malim i velikim slovima, postoji još jedna bitna razlika kod korišćenja komponenti u šablonima: simbolička imena za istu klasu (pa samim tim istu komponentu) mogu da se razlikuje među upotrebama u različitim komponenatama.
Drugim rečima, ime koje developer da simbolu prilikom registracije je ono pomoću koga komponenta može da se deklariše u šablonu.
Na primer, klasa se može preimenovati alijasovanim importom; pored toga, klasa uopšte ne mora da ima ime ukoliko je iz modula izvezena kao podrazumevani simbol, već joj se simboličko ime dodeljuje tek prilikom uvoza.

Registrovana komponenta se u šablon može postaviti kao bilo koji drugi HTML element.
Pošto ne može da ima decu, može se koristiti i samo-zatvarajući tag.

\begin{verbatim}
<Component></Component>
<Component/>
\end{verbatim}

\subsection{Ulazi}

Da bi komponentu bilo moguće parametrizovati, uvodi se pojam  \textbf{ulaza} (\textsl{input}) u komponentu.
Ovaj pojam je pandam svojstvima koje imaju HTML elementi.

Ulazi u komponentu se definišu deklarisanjem javnog svojstva na klasi.

Pošto se za proveru ispravnosti tipova u izvornom kodu koristi strogi režim Tajpskripta, ulazima za koje se očekuje da uvek dobiju vrednost iz šablona u kome se koriste dodaje se \code{!} -- stoga, ovo označava \textbf{obavezni ulaz}, pa Vejn prijavljuje grešku ukoliko mu se vrednost ne prosledi.
Za bi se definisala podrazumevana vrednost ulaza, dovoljno je da se svojstvo jednostavno inicijalizuje.
Ulazi sa podrazumevanom vrednošću ne mogu biti obavezni.

\begin{verbatim}
@Template(`...`)
class ExampleComponent {
  public input1!: number
  public input2 = 2
  input3: any
  private prop1: any
}
\end{verbatim}

U prethodnom primeru definisani su, redom: eksplicitno definisan obavezni ulaz; eksplicitno definisan opcioni ulaz sa podrazumevanom vrednošću \code{2}; implicitno definisan ulaz; obično svojstvo na klasi koje se može iskoristiti za čuvanje stanja komponente.

U šablonima, ulazi se koriste kao da je u pitanju bilo koji HTML atribut, s tim što se ime ulaza omeđuje parom uglastih zagrada, a umesto konkretne vrednosti se očekuje \textbf{vezivna sintaksa}.

Osim vezivanja za svojstva klase, veza se može ostvariti i sa alijasima, ukoliko se u šablonu koriste direktive koje ih stvaraju.
Više reči o tome u odeljku o \code{w:for} direktivi.

\subsection{Izlazi}

HTML elementi imaju događaje, a Vejn komponente imaju \textbf{izlaze} (\textsl{output}).
Izlaz se deklariše kao metoda bez tela u telu klase.
Da bi se izlaz komponente osluškivao, dodaje se HTML atribut, pri čemu je ime izlaza omeđeno parom oblih zagrada, a umesto prosleđivanja konkretne vrednosti se očekuje \textbf{poziv funkcije}, tj. metode definisane u klasi.

Da bi se pribavila referenca na vrednosti koje se emituju kroz izlaz, u šablonu se koristi specijalni znak \code{\#} kao argument funkcije.
Pored njega, argumenti funkcije se mogu referencirati na svojstva klase ili mogu biti literali.

\section{Direktive}

Direktive su posebne naredbe koje se umeću u šablone u vidu HTML tagova.
Primaju proizvoljan broj čvorova za decu i služe za parametrizaciju načina na koji će se taj sadržaj prikazati u rezultujućem DOM-u.

Sintaksno, direktive se od HTML elemenata i Vejn komponenti razlikuju po prefiksu \code{w:}.
Zbog svoje prirode, parametri se -- umesto navođenjem atributa u vidu parov ključ-vrednost -- definišu potpuno proizvoljnom sintaksom koja se piše u nastavku imena, a u sklopu otvarajućeg taga.

Developer nema mogućnost da definiše proizvoljne direktive; sve direktive su sastavni deo Vejna.

\subsection{Uslovi}

Prisustvo ili odsustvo dela šablona može se kontrolisati korišćenjem direktive \code{w:if}.
Uslov na osnovu koga se određuje da li će se grupa elemenata naći u DOM-u ili ne se navodi u vidu izraza vezivne sintakse.

\begin{verbatim}
<w:if isLoggedIn>
  Welcome, {{ user.name }}!
  <img [src]="user.image" [alt]="user.name"/>
</w:if>
\end{verbatim}

U prethodnom primeru se, u zavinosti od trenutne vrednosti dodeljene svojsvu klase \code{isLoggedIn}, isečak šablona unutar direktive \code{w:if} ili prikazuje ili ne.

\subsection{Nizovi}

Predstavnik struktura podataka za rad sa kolekcijama istorodnih elemenata u Javaskriptu su nizovi, implementirani pomoću prototipa \code{Array}.
U šablonima se može vršiti iteracija nad nizovima korišćenjem direktive \code{w:for}.

U nastavku se navodi sintaksa oblika \code{(item, index) of items; key: id}, gde je

\begin{itemize}
\item \code{items} -- svojstvo ili lanac pristupa sa klase,
\item \code{item} -- simboličko ime za jedan element iz niza \code{items},
\item \code{index} -- simboličko ime za indeks elementa \code{item} iz niza \code{items},
\item \code{id} -- lanac pristupa kojim se definiše način na koji se može preuzeti \textquote{ključ} po kome se semantički razlikuju elementi niza,
\end{itemize}

dok su \code{of}, \code{;} i \code{key:} ključne reči kojima se poboljšava čitljivost sintakse.

Simbočko ime za indeks elementa nije obavezno navesti; u slučaju da se ono ne navede, zagrade koje omeđuju simboličko ime za vrednost elementa niza nisu obavezne.

Deo šablona koji je naveden unutar direktive čini šablon koji će se koristiti za kreiranje DOM-a.
Ovaj deo šablona parametrizovan je vrednošću elementa koji je trenutno posećen tokom iteracije (\code{item}) i, opciono, njegovom brojnom indeksu (\code{index}).
U direktivi, ova imena postaju deo opsega koji se može koristiti za vezivnu sintaksu u šablonima (pored svojstava i metoda klase).

\begin{verbatim}
<ol>
  <w:for user of users; key: id>
    <li>{{ user.name }} ({{ user.score }})</li>
  </w:for>
</ol>
\end{verbatim}

\subsection{Skraćeni oblik}

Vrlo je čest slučaj da se unutar direktive nađe samo jedan HTML element ili samo jedna komponenta.
U tim slučajevima, način zapisa direktive se može skratiti tako što se direktiva navodi kao HTML atribut elementa ili komponente, pri čemu se za ključ koristi ime direktive (sa prefiksom \code{w:}), a kao vrednost se koristi sintaksa koju ta direktiva razume.

\begin{verbatim}
<span [w:if]="!isLoggedIn">Welcome, guest!</span>

<ol>
  <li [w:for]="user of users; key: id">
    {{ user.name }} ({{ user.score }})
  </li>
</ol>
\end{verbatim}

Dve direktive se ne mogu istovremeno naći na jednom elementu ili jednoj komponenti.
Ukoliko se javi potreba za ovakvom strukturom, barem jedna od direktiva se mora zapisati u regularnoj (dužoj) sintaksi.

\section{Stilovi}

Kada se govori o korisničkim interfejsima, stilovi su nezaobilazna tema.
U brauzerima se oni definišu pomoću CSS-a.

\subsection{Enkapsulacija}

Tradicionalno, CSS se primenjuje globalno, nad celim dokumentom.
Kako je osnovna jedinica izgradnje korisničkog interfejsa komponenta, globalni stilovi nisu idealno rešenje jer se onda mora voditi računa o tome da stil jedne komponente ne utiče na stil druge, što remeti mentalni model kojim se komponenta proglašava kao nezavisna celina.

Ovo nije novi problem -- uviđen je početkom ere jednostraničnih aplikacija -- i počelo se sa radom na njegovom zvaničnom rešenju koje treba da postane deo standarda.
Rešenje je nazvano \textquote{DOM u senci} (\textsl{Shadow DOM}).
Specifikacija \textquote{opisuje metodu kombinovanja više DOM stabala u jednu hijerarhiju i način na koji ova stabla međusobno interaguju u okviru dokumenta, omogućujući na taj način bolju kompoziciju DOM-a \cite{w3c:shadow-dom}}.
Deo specifikacije koji se tiče stilova a povezan je sa DOM-om u senci naziva se \textsl{CSS Scoping Module}.
Specifikacija još uvek nije finalizirana, ali moderni brauzeri nude eksperimentalnu podršku.

Pozajmljujući ovu ideju, ali ne oslanjajući se na DOM u senci zbog eksperimentalne podrške u brauzerima, stilovi definisani u okviru Vejn komponente će biti vezani isključivo za nju -- ovaj pojam se naziva \textbf{enkapsulacija stilova}.
Stilovi su enkapsulirani i tiču se isključivo dela DOM-a za koji su vezani.

Na primer, ukoliko se u jednoj komponenti promeni boja elementima koji odgovaraju selektoru \code{button}, a u drugoj se promeni veličina teksta koristeći isti selektor, stilovi će biti primenjeni samo u šablonima komponenti za koje su stilovi definisani -- drugim rečima, stilovi neće biti pomešani i nijedno dugme na stranici, bilo unutar tih komponenti ili izvan, neće dobiti mešavinu ova dva stila.

Stilovi se definišu u okviru string literala koji se prosleđuje kao prvi i jedini argument dekoratoru \code{@Style} koji treba primeniti nad klasom koja opisuje komponentu kojoj se vezuju stilovi.

\begin{verbatim}
@Template(`<button>Red</button>`)
@Style{`button { color: red; }`}
class RedButton { }

@Template(`<button>Large</button>`)
@Style{`button { font-size: 2em; }`}
class LargeButton { }
\end{verbatim}

\subsection{Nadogradnja}

Standardni način definisanja CSS-a je proširen posebnim selektorom \code{:host}.
Ime i semantika selektora je pozajmljena iz \cite[\S3.2]{w3c:css-scoping-1}, ali je način funkcionisanja emuliran (kao i sama enkapsulacija stilova).

CSS se odnosi na \emph{sadržaj} komponente -- bez samog elementa koji predstavlja komponentu, u koji je smešten njen sadržaj.
Jedini način da se ovom elementu pristupilo \textquote{iznutra} (iz koda same komponente) je pomoću selektora \code{:host}.

\begin{verbatim}
:host {
  background-color: red;
}
\end{verbatim}

Osim ovoga, podržano je korišćenje registrovanih imena komponenti za selektore (iako se takvo ime -- sa velikim početnim slovom -- neće odraziti u DOM-u).

\begin{verbatim}
@Register(Child1, Child2)
@Template(`<Child1/> <Child2/>`)
@Style(`Child1 { color: red; }
        Child2 { font-siye: 2em; }`)
class Example { }
\end{verbatim}

% \subsection{Specijalni slučaj}

% TODO: lobotomizovana sova ide u delu gde opisujemo KAKO stvari rade

\subsection{SCSS}

Danas se korišćenje nekog alata koji olakšava pisanje CSS-a podrazumeva.
Najpopularniji su preprocesori, a kao vodeći se izdvaja \textbf{SCSS}, koji se koristi više od pisanja čistog CSS-a~\cite{sojs:2017}.

CSS preprocesor je program koji generiše CSS na osnovu sintakse preprocesora.
Pojavilo se mnogo ovakvih alata, i većina dodaje novine koje ne postoje u CSS-u: miksini, ugnježđavanje pravila, nasleđivanje, itd.
Imaju za cilj da definicije stilova učine čitljivijim i jednostavijim za održavanje~\cite{mdn:glossary:css-preprocessor}.

Stilovi u Vejnu se, umesto u CSS-u, pišu u SCSS-u.
Ova opcija se ne može isključiti.
Pošto je SCSS striktni nadskup CSS-a, ukoliko developer ne želi da koristi SCSS -- jednostano ne treba da iskoristi nijedan njegov deo i kod će biti potpuno kompatibilan sa običnim CSS-om.

\section{Pokretanje}

Pored frejmvorka, deo Vejna je CLI koji je primarni način za generisanje izvršne verzije koda, spremne za produkciju.
Paket je objavljen tako da se iz komandne linije može koristiti izvršni modul pod imenom \code{wane}.

\subsection{Produkcija}

Jedna od dve dostupne komande je komanda \code{build}.

\begin{verbatim}
$ yarn wane build
\end{verbatim}

Ova komanda pokreće kompilator.
Ukoliko aplikacija prođe proces kompilacije uspešno, na standardni izlaz se ispisuju informacije o veličini izlaznih datoteka, a u \code{dist} folderu se pojavljuju datoteke spremne da bude prebačene na server.

Proces kompilacije izvornog koda se može kontrolisati pomoću konfiguracione datoteke.
Ova datoteka treba biti smeštena u korenom direktorijumu Vejn projekta, sačuvana pod imenom \code{wane.toml}.
Sadržaj datoteke treba da bude validna TOML sintaksa\footnote{TOML (\textsl{Tom's Obvious Minimal Language}) je jednostavan format namenjen za konfiguracione datoteke koji je lako čitljiv ljudima zbog pregledne sintakse i očigledne semantike. Nastao kao kontrast JSON-u koji je namenjen mašinama i YAML-u koji je često previše kompleksno rešenje \cite{toml:spec}.}, pri čemu su dozvoljena sledeća dva polja.

\begin{itemize}
\item \code{output.build} je relativna putanja do direktorijuma u kojem treba da budu smeštene izlazne datoteke.
  Ukoliko direktorijum ne postoji, biće napravljen.
  Ukoliko postoji, njegov sadržaj će biti obrisan.
  Ako vrednost nije podešena, podrazumeva se \code{dist}.
\item \code{debug.pretty} je fleg kojim se kontroliše da li izlazne datoteke treba da prođu proces minifikacije.
  Na šta se tačno odnosi \textquote{minifikacija} nije definisano i nije deo javnog API-ja.
  Ovime je omogućeno da se ovaj proces unapređuje u narednim verzijama Vejna bez poterbe da developer čini promene u kodu.
  Služi za debagiranje izlaznog koda.
  Ako vrednost nije podešena, podrazumeva se \code{false}.
\end{itemize}

\subsection{Razviće}

Tokom razlića aplikacije može se koristiti posebna komanda \code{start}.
Ovime će biti automatski kreiran dev-server, a \code{src} direktorijum u kome se nalazi izvorni kod će se osluškivati za promene.
Svaki put kada se promena detektuje, proces kompilacije će biti pokrenut iznova, a stranica u brauzeru će biti osvežena kako bi developer mogao da vidi izmene koje su načinjene.

\section{Primer aplikacije}

% TODO nešto jednostavno što pokazuje sve
