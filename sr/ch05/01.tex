\section{Proces analize koda}

Analiza koda počinje od podrazumevanog izvezenog simbola datoteke na putanji \code{src/index.ts}.
Ukoliko se ne radi o klasi, ukoliko klasa nije obeležena dekoratorom \code{@Template} ili ukoliko je prosleđen šablon nevalidan, faza analize se prekida.

\subsection{Stablo analizatora fabrika}

Prvi korak u analizi je konstrukcija stabla analizatora fabrika komponenti i direktiva.
\textbf{Fabrika} (\textsl{factory}) je funkcija koju generator treba da ispiše; ona treba da se koristi tokom izvršenja aplikacije kako bi se kreirala instanca komponente ili direktive.
U fazi kompilacije, svaka fabrika je predstavljena svojim \textbf{analizatorom} (\textsl{factory analyzer}).
Analizatori sagledavaju kod komponente, poziciju komponentu u stablu, njen API i druge osobine kako bi generator koda mogao da iskoristi ove informacije, na osnovu kojih će biti u mogućnosti da generiše kod u skladu sa namerom developera.

U ovoj fazi analize je važno uspostaviti međusobni odnos komponenata i direktiva, ali i načina na koji su one povezane putem šablona.
Na primer, nije dovoljno znati da komponenta~A ima za dete komponentu~B; važno je dobiti informaciju o tome gde se u \emph{pogledu} komponente~A nalazi mesto za koje je vezana komponenta~B.

\textbf{Pogled} (\textsl{view}) komponente ili direktive je deo šablona kojim ona rukovodi.

Mada struktura stabla koje čini pogled komponente može biti ekvivalentna strukturi samog šablona komponente, ovo nije slučaj ukoliko se u šablonu komponente nalaze direktive.
U tom slučaju se pogled komponente završava na mestu gde počinje direktiva, a deo šablona koji se nalazi u unutrašnjosti direktive počinje obradu kao novi šablon.
Pored moguće razlike u strukturi, pogled nosi sa sobom mnogo više informacija nego šablon.
Može se reći da je šablon struktura koju developer definiše kako bi \textquote{objasnio} kompilatoru kako da kreira pogled.

Komponente se prepoznaju po velikom početnom slovu, direktive po prefiksu~\code{w:}.
Za ostalo se podrazumeva da se radi o HTML elementima, što znači da su pored postojećih HTML elemenata podržane i veb-komponente.
Nailazak na HTML elemente se samo evidentira kao deo pogleda komponente ili direktive koja se trenutno obrađuje.
S druge strane, kada se naiđe na HTML tag koji se prepozna kao komponenta ili direktiva, započinje se rekurzivni proces izgradnje stabla pogleda za novu komponentu ili direktivu.
Po okončenju procesa, dobijeni analizator se beleži kao dete u analizatoru nad kojim se trenutno vrši obrada.

Sem komponenti i direktiva, \textbf{tekst} i \textbf{interpolacija} su takođe vrste čvorova koje se nalaze u stablu pogleda.

\begin{verbatim}
@Template(`<div id="a"></div>`)
class A { }

@Template(`<div id="b"></div>`)
class B { }

@Register(A, B)
@Template(`
  <div id="c"></div>
  <w:if cond>
    <A/>
  </w:if>
  <B/>
`)
export default class C { }
\end{verbatim}

U primeru TODO, ulazna komponenta je~\code C; od nje počinje izgradnja stabla.
Njenom pogledu pripadaju \code{div\#c}, uslovna direktiva \code{w:if:cond} kao struktura bez unutrašnjeg sadržaja i komponenta~\code{B} bez unutrašnjeg sadržaja.
Struktura koja opsuje decu analizatora fabrike komponente~\code C definiše mapiranje čvora \code{w:if:cond} na analizator fabrike uslovne direkive, a čvora~\code{B} na analizator fabrike komponente~\code{B}, koji ima pogled iste strukture kao i šablon.

Pogled analizatora fabrike uslovne direktive \code{w:if:cond} ima samo jedan čvor, i to komponentu~\code A.
Slično komponenti~\code B iz šablona komponente~\code C, ovaj čvor pokazuje na analizator fabrike komponente~\code A.

Tokom sagledavanja šablona radi generisanja pogleda, za svaki čvor se beleži u kojoj je fabrici \textbf{definisan} (\textsl{definition factory}) i koja je fabrika \textbf{odgovorna} za njega (\textsl{responsible factory}).
Fabrika u kojoj je čvor definisan je ona fabrika u čijem se \textit{šablonu} nalazi čvor.
Odgovorna fabrika je fabrika u čijem se \textit{pogledu} čvor nalazi.

Čvorovi stabla koji sačinjavaju pogled nose sa sobom informaciju o svom tipu i o \textbf{povezima} (\textsl{binding}) koji su nad njim definisani.
Povezi ukazuju na način na koji je postignuta veza između modela i šablona.
Na primer, \code{<div id="a" [className]="b"{}>} je definicija čvora koji predstavlja HTML element sa dva poveza.
Prvi povez se odnosi na svojstvo elementa pod nazivom~\code{id} i povezan je sa \textit{literalom}~\code{a} koji je tipa string.
Ova inforamcija je implicitno zaključena i čvor pogleda bi izgledao isto da je umesto \code{id="a"} autor naveo \code{[id]="{}'a'{}"}.
Drugi povez odnosi se na svojstvo~\code{className} i povezan je sa \textit{svojstvom}~\code{b}.

Strukture koje opisuju \emph{vrednosti} u povezima (literale, svojstva, metode, itd), odnosno \textquote{stranu modela} (spram \textquote{strane šablona} koja je opisana čvorovima) nazivaju se \textbf{povezane vrednosti} (\textsl{bound values}).
One igraju značajnu ulogu u fazi generisanja koda u kojoj se određuje \textit{opseg} iz kog se pribavljaju podaci iz modela.

Proces generisanja stabla analizatora fabrika je jedini proces koji se implicitno pokreće u fazi analize.
Analize u okviru struktura koje su kreirane ovim procesom su \textquote{lenje}, tj. biće pokrenute samo ako informaciju o tome zatraži generator ili sam proces generisanja stabla analizatora fabrika.
Rezultati analize se keširaju pa ne postoji problem sa višestrukim pozivom iste funkcije, ma koliko ona bila komplikovana.

\subsection{Analiza toka koda}

Većina informacija koju analizatori fabrika mogu da pruže baziraju se na rezultatima analize toka koda.
\textbf{Analiza toka koda} je tehnika statičke analize kojom se određuje zamišljena putanja kojom se prolazi prilikom izvršenja koda.
Jedan od načina za analizu toka koda, koji se ovde koristi, naziva se \textbf{apstraktna interpretacija}.
Ovom tehnikom se kod izvršava parcijalno -- samo onoliko koliko je dovoljno da bi se utvrdila osobina koja koja je od interesa.

U ovom odeljku biće definisani neki značajni pomoćni algoritmi koji se koriste u analizi koda komponenti.

Algoritmi se oslanjaju na obilazak i manipulaciju apstraktnog sintaksnog stabla (AST, \textsl{abstract syntax tree}) koda napisanog u Tajpskriptu.
Mada se sintaksa programa predstavlja strukturom stabla, tok izvršenja se predstavlja proizvoljnim grafom.
Na primer, metoda~$A$ može pozivati metodu~$B$, a metoda~$B$ metodu~$A$; metoda može pozivati i samu sebe.
Zato se za obilazak toka koriste algoritmi za obilazak grafova.
Pošto su veze između čvorova neoznačene, tj. ne postoji mera distance, svi algoritmi koriste jednostavne obilaske grafova kao što su pretraga po dubini (DFS, \textsl{depth-first search}) i pretraga po širini (BFS, \textsl{breadth-first search}).

\subsubsection{Pribavljanje tela metoda koje se pozivaju iz bloka}\label{sec:algo:get-bodies-called-from}

Da bi kontrola toka mogla da se prati neometano kroz pozive metoda u okviru klase, neophodno je poznavati koje sve funkcije i metode mogu da se pozove iz jednog bloka koda.
Blok koda najčešće predstavlja metodu ili funkciju, mada to može biti i blok kao što je jedna od grana u uslovnom grananju korišćenjem izraza~\code{if}.

U bloku za koji se pribavljaju tela vrši se iteracija nad svim potomcima sintaksne vrste \code{CallExpression} (\cref{sec:sk:call-expression}).
Među izrazima se traže oni kod koji izraz koji se poziva pripada vrsti \code{PropertyAccessExpression} (\cref{sec:sk:property-access-expression}), i to takav da se na levoj strani nalazi ključna reč \code{this} (\cref{sec:sk:this-keyword}).

Za izraze koji zadovoljavaju ove uslove beleži se ime pozvane metode.
Vodeći se ovim imenom, pronalaze se tela tih metoda.
Ona se dodaju u skup rezultata, a od njih proces počinje da se ponavlja rekurzivno.

\subsubsection{Pribavljanje imena svojstava kojima se vrednost može promeniti izvršenjem bloka}\label{sec:algo:get-prop-names-which-can-be-modified-by}

Kako bi se uvidela veza između modela i šablona, potrebno je razumevanje toga koje metode i kako mogu da utiču na promenu vrednosti svojstva u klasi.

Ovaj algoritam se odvija kroz dve faze.
U prvoj fazi se traže imena svojstava kojima se vrednost može promeniti \emph{direktnim} izvršenjem bloka.
Drugim rečima, ne prolazi se kroz pozive metoda.
Nakon što se prva daza okonča, postupak se rekurzivno ponavlja za sve metode koje se mogu pozvati iz bloka, korišćenjem algoritma opisanog u \cref{sec:algo:get-bodies-called-from}.

Potraga za svojstvima obavlja se prolaskom kroz potomke sintaksne vrste \code{ExpressionStatement} (\cref{sec:sk:expression-statement}).
Od interesa su binarni, prefiksni unarni i postfiksni unarni izrazi.

Za binarne izraze od interesa su samo oni kod kojih je operator jedan od operatora dodele (\cref{sec:sk:tokeni-dodele}).
Ovime se odstranjuju binarni izrazi -- na primer, \code{a < b}.
Zatim se odstranjuju izrazi kod kojih leva strana binarnog izraza ne predstavlja pristup svojstvu (kao \code{a.b}), a potom se odstranjuju i oni kod kojih leva strana pristupa nije \code{ThisKeyword}.
Preostali pristupi u sebi sadrže ime svojstva kojem se menja vrednost (npr. \code{this.foo}).

Za perfiksne i postfiksne unarne izraze, pretraga je jednostavnija.
Uzimaju se samo oni izrazi kod kojih važi da je operand pristup svojstvu, i to takvih da je prvi pristup ključna reč \code{this}.
Kao i u slučaju binarnih izraza, iz njih se dobija ime svojstva kojem se menja vrednost.

\subsection{Analizatori komponenti}

Osnovne informacije o komponenti, nezavisno od njene konkretne upotrebe u stablu komponenti aplikacije, pribavljaju se iz analizatora komponente.
U procesu izgradnje stabla analizatora fabrika, nailaskom na komponentu se najpre kreira njen analizator, pa se onda on koristi za kreiranje analizatora \emph{fabrike} komponente.

\subsubsection{Ime komponente}

Ime komponente ne igra značajnu ulogu u tokom razvića aplikacije jer se nigde ne koristi.
Umesto imena komponente koriste se simboličko ime kojim je komponenta \emph{registrovana}, a ne \emph{definisana}.
Ipak, \emph{neko} ime se mora dodeliti komponenti kada se ona umetne u DOM tokom izvršenja.

Mada ime komponente može biti samo proizvoljan jedinstven string, radi preglednijeg rezultujućeg DOM-a sa boljom semantikom, Vejn koristi informacije kojima raspolaže da komponenti dodeli ime.

Ako je definisano ime klase, ono se koristi kao ime komponente.
Inače se radi o podrazumevano izvezenom simbolu iz modula, pa se koristi ime modula -- osim u slučaju kada je ime modula \code{index}: tada se koristi ime direktorijma u kome se nalazi modul.
Ovako dobijeno ime komponente se za potrebe imena proizvoljnog HTML taga pretvara u \textit{param-case} oblik.

Međutim, HTML standard nalaže da imena proizvoljno definisanih tagova ne samo da moraju da budu mala slova, već da \emph{moraju} sadržati bar jednu crticu kako bi se obezbedila kompatibilnost sa HTML elementima koji će u standard biti dodati u budućnosti (jer se za njih zna da nikad nemaju nijednu crticu).
Ukoliko se ime komponente sastoji od dve reči, ovo ne predstavlja problem, ali imena od jedne reči neće moći da se koriste kao ime taga.

Zato se, radi konzistentnosti, uvek dodaje prefiks~\code{w-} ispred imena HTML taga.
Na primer, komponenta \code{Item} dobija tag \code{w-item}, dok \code{ListOfItems} dobija tag \code{w-list-of-items}, iako bi ime sadržalo crtice i bez perfiksa.

\subsubsection{Konstante}\label{sec:konstante}

Obilaskom koda se za svako svojstvo klase može odrediti da li se ono može smatrati svojstvom iz koga se podaci samo čitaju, a nikada ne menjaju.
Za ovakvo svojstvo klase se kaže da je \textbf{konstanta}.

Preduslov da se svojstvo prolagsi za konstantu jeste da nije ulaz u komponentu -- ovo se jednostavno određuje ispitivanjem modifikatora kojim je obeleženo svojstvo (bez modifikatora -- implicitno \code{public}, eksplicitno \code{public}, \code{protected}, \code{private}).
Zatim se prolazi kroz sve blokove koji predstavljaju tela metoda klase i pribavljaju se imena svojstava koja se mogu modifikovati izvršenjem tog bloka, na osnovu algoritma opisanog u \cref{sec:algo:get-prop-names-which-can-be-modified-by}.
Ukoliko se ime svojstva za koje se ispituje status konstante ne nađe u tako dobijenom skupu, radi se o konstanti.

Ovo se koristi za optimizaciju opisanu u TODO.

\subsection{Analizatori fabrika}

Analizator fabrike služi da generatoru koda pruži neophodne inforamcije da bi mogao da obavi svoj zadatak.
Nalaze se u čvorovima stabla analizatora fabrika.

\subsubsection{Pribavljanje putanje do druge fabrike} % getPathTo

Između svake dve fabrika postoji najkraća putanja.
Pošto se radi o stablu, najkraća putanja je jedina putanja u kojoj se putanja ne \textquote{vraća unazad}, pa je dovoljno koristiti običan DFS ili BFS algoritam za pronalazak ove putanje.

Putanja je predstavljena nizom susednih čvorova u stablu, onim redom kojim ih treba obići da bi se iz jedne fabrike \textquote{stiglo} u drugu.

Ova funkcionalnost je važna za generisanje koda koji se tiče prenosa podataka iz jednog čvora u drugi (što se događa prilikom vezivanja vrednosti za šablon).

\subsubsection{Pribavljanje stvarnog imena vezane vrednosti} % hasDefinedAndResolvesTo

Autoru je u šablonima prilično intuitivno na koji opseg se odnose imena vezana za šablon, ali proces obračunavanja \emph{stvarne} vrednosti (ili putanje do vrednosti) je drugačiji, pogotovo kada se uzme u obzir činjenica da se na nekim mestima kod opitmizuje (v.~\cref{sec:konstante}).

U slučaju \textbf{komponenti}, pristupni lanac se najpre razbija na delove.
Među svojstvima, geterima i metodama se traži onaj sa istim imenom kao prvi deo lanca.
Ukoliko se on ne pronađe, znači da se u njoj ne nalazi tražena referenca.
Ovo je znak generatoru, koji poziva ovu metodu, da nastavi traženje dalje uz lanac roditelja duž grane u stablu analizatora, sve dok ne naiđe na prvi analizator koji se ne nalazi između definicione fabrike i odgovorne fabrike za čvor u kome je definisan povez (više u TODO).
Ukoliko se u deklaraciji klase pronađe traženi simbol, onda se na osnovu~\cref{sec:konstante} određuje da li se radi o konstanti ili ne.

\begin{lstlisting}
@Template(`This is a /@\tikzmark{view-left}@/{{ value }}/@\tikzmark{view-right}@/.`)
class Component {
  /@\tikzmark{code-left}@/value/@\tikzmark{code-right}@/ = 21
}
\end{lstlisting}
\begin{tikzpicture}[remember picture, overlay]
  \node (view) at ($(view-left)!.5!(view-right)$) {};
  \foreach \x/\y in {view-left/view-right, code-left/code-right} \WrapInBox{\x}{\y}
  \draw [Circle-Stealth] (view) edge [bend left] (code-right);
\end{tikzpicture}

Kod \textbf{uslovnih direktiva}, proces je znatno jednostavaniji jer je \emph{nemoguće} da vrednost bude vezana za nju.
Ova direktiva ne nudi nikakve simbole koje bi šablon mogao da iskoristi za povezivanje.

\textbf{Direktive za ponavljanje}, s druge strane, pružaju jedan ili dva simbola koji se mogu iskoristiti.
To su \textit{iterativna konstanta} (uvek) i \textit{iterativni indeks} (ukoliko se navede).
Kao i u slučaju komponenti, lanac se razbija na delove, ali se poredi samo sa jednom ili dve gorenavedene vrednosti.
Nijedan ni drugi simbol ne mogu biti konstante.
U slučaju pogotka, generatoru se vraća informacija o tome koji iterativni simbol je u pitanju.
Ova vrednost se zatim koristi da se pristupi \emph{pravom} imenu (umesto simboličkom koje developer zadaje radi razumevanja koda).

\subsubsection{Povezi u pogledu i njihovo mapiranje}\label{sec:getPropsBoundToView}

Postoji i inverzni način ispitivanja koji daje sličnu informaciju.
Umesto da se za konkretan lanac pristupa ispituje da li pripada određenom analizatoru, mogu se dobiti sve instance poveza za koje bi pribavljanje stvarnog imena dalo pogodak.

U ovo spadaju i povezi definisni nad \textquote{pravim} HTML elementima i nad komponentama i direktivama koji postoje samo prividno tokom pisanja Vejn aplikacije.
Mogu se dobaviti i particije ovog skupa po ovoj osobini.

\begin{lstlisting}
@Register(B)
@Template(`The value is {{ val1 }} and <B [b]="val2"/>.`)
class A { private val1!: any; private val2!: any }
\end{lstlisting}

U gornjem primeru, za komponentu~\code{A}, to su

\begin{itemize}
  \item povez definisan inteprolacijom \code{\{\{~val1~\}\}} (interpolacija se obavlja nad \code{Text} čvorom koji je deo HTML-a) i
  \item ulaz \code{ba]="val2"} (primenjen nad komponentom~\code B što je Vejn komponenta).
\end{itemize}

\subsubsection{Mape razlika}\label{sec:getDomDiffMap}\label{sec:getFaDiffMap}

Od važnosti nije samo \emph{koji} povezi postoje, već i na kojim se elementima nalaze i koja je fabrika odgovorna za njihovo ažuriranje.
U ovu svrhu definišu se mape razlika.
Mape razlika sadrže preslikavanje elementa sistema koji se ažurira u niz poveza koje na njima treba ažurirati.
Zovu se \textquote{mape razlika} jer pokazuju kako treba \emph{mapirati} dobijene \emph{razlike} prilikom pokretanja procesa ažuriranja unutar fabrike.
Postoje \textbf{mape razlika u DOM-u} i \textbf{mape razlika u fabrikama}.
Razlikuju se u tome o kom je elementu sistema reč -- o postojećim HTML elementima ili o elementima specifičnim za Vejn (tj. o komponentama i direktivama).

\begin{lstlisting}
@Template(`
  <div [id]="a" [className]="b">{{ a }}</div>
  <w:for foo of foos>
    <span [className]="a">{{ foo }}</span>
  </w:if>
`)
class Component {
  a!: any
  b!: any
  foos!: any[]
}
\end{lstlisting}

Na primer, u TODO isečku koda definisana je komponenta čija bi mapa razlika u DOM-u definisala sledeće preslikavanje.

\begin{itemize}
  \item Za čvor \code{<div>}: \code{[id]="a"} i \code{[className]="b"}.
  \item Za tekstualni čvor u \code{div}-u: \code{\{\{ a \}\}}.
  \item Za čvor \code{<span>} unutar uslovne direktive: \code{[className]="a"}.
\end{itemize}

Za razliku od \code{[className]="a"}, tekstualni čvor \code{\{\{ b \}\}} unutar \code{span}-a nije deo preslikavanja -- simbol \code{foo} nije definisan u komponenti \code{Component}, već u direktivi \code{w:for} kao alijas za trenutni element niza \code{foos} prilikom iteracije.

S druge strane, mapa razlika u fabrikama bi sadržala samo jedno preslikavanje.

\begin{itemize}
  \item Za dikretivu \code{w:if}: svojstvo \code{a} vezano kao uslovna promenljiva.
\end{itemize}

\subsubsection{Indeksi čvorova u šablonima i pogledima}\label{sec:getIndexesFor}\label{sec:getSingleIndexFor}

Mape indeksa bile bi beskorisne za generisanje koda ukoliko ne bi bilo poznato kako doći do odgovarajućih instanci tokom izvršenja programa.
Pošto postoji jasno definisano uređenje čvorova, svakom je dodeljen indeks u skladu sa BFS obilaskom.
Za čvor u šablonu ili pogledu se može pribaviti odgovarajući indeks u slučaju komponente ili odgovorajući indeksi u slučaju direktive.

Direktive imaju dva indeksa jer su u realnom DOM-u predstavljene sa dva HTML čvora -- tačnije, dva komentara.
Više u TODO.

\subsubsection{Povezi definisani nad sidrom}\label{sec:getSelfBindings}

\textbf{Sidro} (\textsl{anchor}) je čvor pogleda roditeljskom čvoru iz stabla analizatora zbog koje se analizator pojavio u stablu.
Drugim rečima, sidro je inverzno mapiranje čvora pogleda na dete-analizator.
Sidro se koristi da bi se pristupilo povezima definisanim nad komponentom ili direktivom.
U sledećem primeru, sidro za komponentu~\code B je čvor~\code{<B/>} definisan u komponenti~\code A.

\begin{lstlisting}
@Template(`{{ b }}`)
class B { public b!: any }

@Register(A)
@Template(`<B [b]="val"/>`)
class A { private val!: any }
\end{lstlisting}

Analizator sa sidra može dovući listu svih poveza koji tu deklarisani.
U prethodnom primeru, to je ulaz \code{[b]="val"}.

Jedna od primena ovoga jeste pronalazak toga koji izlazi se koriste nad komponentom i koju metodu poziva u komponenti u kojoj je sidro definisano (v. TODO).

\subsubsection{Određivanje fabrika na koje se može uticati izvršenjem bloka} % getFactoriesAffectedByCalling
\label{fn:getFactoriesAffectedByCalling}

Osim određivanja metoda koje se mogu pozvati unutar jedne komponente, važno je odrediti i koje metode se mogu pozvati \emph{izvan} nje.
Jedini način da metoda komponente pozove metodu druge komponente jeste ukoliko su komponente povezane izlazom.

Nakon što se na osnovu algoritma~\cref{sec:algo:get-bodies-called-from} pronađu metode koje mogu biti pozvane određenim blokom, traže se one metode koje su obeležene kao javne, jer to znači da se mogu iskoristiti kao izlazi.
Zatim se pomoću~\cref{sec:getSelfBindings} filtriraju metode koje zaista jesu iskorišćene kao izlaz, i pamte se imena metoda koja se pozivaju u komponenti-pretku.

Za ovako dobijene metode traži se koja se unija svojstava koji se mogu promeniti u toj komponenti (na osnovu~\cref{sec:algo:get-prop-names-which-can-be-modified-by}).
Za istu komponentu se traže i sva svojstva i/ili geteri definisani u klasi koja je upisuje (na osnovu~\cref{sec:getPropsBoundToView}).
Ukoliko postoji presek između ova dva skupa, znači da se pozivom metode iz originalne komponente može javiti promena na pogledu roditeljske komponente.
Ovo generator koda koristi kako bi se odredilo za koje fabrike treba pozvati metode ažuriranja (v. \cref{subsubsec:pretplata-na-dogadjaje-iz-doma}).

Postupak se zatim rekurzivno nastavlja, jer se iz komponente-pretka mogu dalje pozivati metode koje predstavljaju izlaze, koje su u nekom daljem pretku vezane za neku drugu metodu, itd.
Po završetku obilaska, dobija se skup svih fabrika na čije poglede jedna metoda može uticati.

% TODO: primer
