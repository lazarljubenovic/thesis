\section{Proces generisanja koda}

U fazi generisanja koda izdvaja se pet glavnih celina:

\begin{itemize}
  \item priprema,
  \item preprocesiranje komponenti,
  \item generisanje fabrika,
  \item postprocesiranje komponenti i
  \item bandlovanje.
\end{itemize}

\subsection{Priprema}

U fazi pripreme generišu se neki generički fajlovi sa pomoćnim funkcijama koje se mogu koristiti tokom ostalih faza generisanja koda.

\subsection{Preprocesiranje komponenti}

U fazi preprocesiranja komponenti menja se k\^od klase napisan od strane autora.
Pored vršenja transformacija koda, informacije o promenama dostavljaju se analizatoru kako bi dopunile dostupne informacije.

Obavljene transformacije tiču se \textbf{asinhronih} delova koda.
Naime, najčešće se promene u modelu dešavaju kao posledica pokretanja neke akcije korisnika tokom njegovog interagovanja sa korisničkim interfejsom.
Međutim, postoje još dva inicijatora: tajmeri i mreža.

U Javaskriptu se tajmeri postavljaju korišćenjem globalno dostupnih funk\-ci\-ja \code{set\-Time\-out} i~\code{set\-In\-ter\-val}.
Mrežni zahtevi se mogu obaviti pomoću \code{XML\-Http\-Request}-a ili \code{fetch}-a.

\subsubsection{Ubrizgavanje fabrike}
\label{subsubsec:ubrizgavanje-fabrike}

Da bi se omogućilo da se pogled ažurira nakon izvršenja asinhronog dela koda, potrebno je da se pozove metoda koja vrši njegovo osvežavanje.
Međutim, metoda za osvežavanje nalazi se u fabrici komponente, a klasa nema pristup njoj.
Zato je prvo neophodno da se obezbedi da referenca postoji.

Referenca se prosleđuje kroz konstruktor.
Ukoliko on postoji, dodaje se novi argument; inače se kreira konstruktor sa praznim telom i argumentom kojim se čuva referenca na fabriku.

\begin{lstlisting}
// pre
class Foo { }

// posle
class Foo { 
  constructor (private factory)
}
\end{lstlisting}

\subsubsection{Normalizacija i umetanje koda}

Obilaskom AST-a klase kojom je deklarisana komponenta traže se instance poziva funkcije \code{setTimeout} i funkcije \code{setInterval}, kao i pojave poziva metoda \code{.then} i \code{.catch} (nad instancama klase \code{Promise}).
Sve ove funkcije i metode kao prvi argument primaju funkciju koju treba izvršiti kada se dogodi odgovarajući asinhroni događaj.

Postoji nekoliko načina na koji se može proslediti ova funkcija.
To može biti referenca na funkciju, a funkcija se može i direktno deklarisati; pri tome za drugi slučaj postoje dva načina: korišćenjem ključne reči \code{function} ili kao \textquote{streličastu} funckiju, korišćenjem tokena \code{=>}.
Streličasta funkcija takođe ima više varijanti -- zagrade oko argumenata su opcione ako postoji samo jedan argument (a nije mu specificiran Tajpskript tip), a zagrade za telo funkcije i ključna reč \code{return} su takođe opcione u nekim slučajevima.

% TODO
% Primer

Zbog velikog broja načina na koji se funkcija može definisati, pre nego što se u kod umetne poziv funkcije za ažuriranje pogleda, vrši se \textbf{normalizacija funkcije}.
Funkcija se transformiše tako da se radi o streličastoj funkciji (ovime se izbegavaju greške ukoliko autorova funkcija koristi \code{this} koje treba da pokazuje na klasu, a ne na funkciju kreiranu ovom transformaciju), a autorov kod se pretvara u strukturu nalik na IIFE\footnote{\textsl{Immediately Invoked Function Expression}, često korišćeni obrazac u Javaskriptu gde se funkcija koja se definiše odmah i izvršava; na primer, \code{(() => 21)()}.}.
Ovime je dobijen ekvivalentan kod, ali je napisan u obliku u kome je lako dodati sporedne efekte koji se neće mešati sa namerom koju ima autor funkcije.

\begin{lstlisting}
(...args: any[]) => {
  const result = (/* autorov kod */)(...args)
  /* umetnut kod */
  return result
}
\end{lstlisting}

Svaki asinhroni deo koda se obeležava nenegativnim celim brojem koji predstavlja indeks elementa u nizu.
Umetnut kod poziva metodu iz fabrike koja se bavi ažuriranjem pogleda, pri čemu se pomenuti broj koristi za pristup potrebnoj metodi.

\begin{lstlisting}
/* umetnut kod */
this.factory.updateAsync[0]()
\end{lstlisting}

\subsection{Generisanje fabrika}

Fabrike se generišu u posebnim datotekama, obilaskom stabla analizatora fabrika.
U ovoj fazi čvorovi više nisu međusobno zavisni, pa redosled obilaska nije od važnosti.
Za svaki tip analizatora fabrika postoji odgovarajući generator koda koji se koristi.

O samoj strukturi fabrika govori se u \cref{sec:anatomija-fabrika-komponenti}, \cref{sec:optimizacija-fabrika-komponenti}, \cref{sec:fabrike-uslovne-direktive} i \cref{sec:fabrike-direktive-za-ponavljanje}.

\subsection{Postprocesiranje komponenti}

U fazi postprocesiranja, autorov kod se dodatno obrađuje kako bi se pripremio za krajnji produkt.

\subsubsection{Uklanjanje dekoratora}

Kada autor koristi dekoratore u kodu, oni služe za kompilatoru dostave neke specifične informacije kako bi se u fazi generisanja fabrika generisao odgovarajući kod.
Sami dekoratori nemaju nikakav uticaj na izvršenje koda.

Zapravo, kada autor uveze simbole kao \code{Template} iz modula \code{wane}, uvoze se samo Tajpskript deklaracije.
Kod dekoratora koji bi se izvršavao kada se aplikacija pokrene uopšte \textbf{ne postoji}.
Ako bi se kod pokrenuo kao takav, došlo bi do greške \code{ReferenceError} jer se simboli kao \code{Template} ne bi pronašli.
Zato se svi dekoratori u potpunosti brišu.

\subsubsection{Izvoz klasa}

Svaka klasa definisana od strane autora prebacuje se u posebnu datoteku koja se uvozi u odgovarajuće fabrike.
Da bi se obezbedilo da je uvoz moguć, osim pukog kopiranja teksta koji čini deklaraciju klase, mora se dodati i ključna reč \code{export}, ukoliko ona ne postoji.

\subsubsection{Uklanjanje konstanti}

Na osnovu rezultata algoritma opisanog u \cref{sec:konstante}, poznato je koja se svojstva klase mogu ukloniti iz deklaracije klase i prebaciti u datoteku u kojoj se nalaze sve konstante.

Pored pukog uklanjanja svojstava sa klase, važno je obraditi i delove koda gde se konstanta koristi i uvesti modul sa konstantama kako bi se njoj moglo pristupiti.

\begin{lstlisting}
// pre 
class Component {
  private constant = 21
  private variable = 42
  private get sum () {
    return this.constant + this.variable
  }
  private mutate (val) {
    this.variable = val
  }
}

// posle
Component$constant = 21
class Component {
  private variable = 42
  private get sum () {
    return Component$constant + this.varible
  }
  private mutate (val) {
    this.variable = val
  }
}
\end{lstlisting}

\subsection{Bandlovanje}\label{subsec:bandlovanje}

Nakon što se sve datoteke generišu, sledi vaza bandlovanja.
Kao i ceo proces generisanja koda, i ovo je implementacioni detalj i ne treba da se tiče autora koda.
Trenutno se koristi Rolap (\textsl{Rollup}) za sklapanje svih datoteka u jednu, ali se ovaj izbor može promeniti u bilo kom trenutku, i k\^od autor\=a neće morati da pretrpi nikakve promene, čak ni u konfiguracionim podešavanjima.

U toku bandlovanja vrši se i minifikacija.
Za minifikaciju koristi se Terzer (\textsl{terser})\footnote{Ranije Aglifaj-E-Es (\textsl{UglifyES}), najpopularniji alat za minifkaciju Javaskript modula.} u vidu plagina za Rolap.

Kako bi se obezbedilo bolje menglovanje (\textsl{mangling}) imena svojstava u kodu (što nije uvek sigurna operacija prilikom minifikacije), sva interna imena koja Vejn generiše dobijaju prefiks \code{\_\_wane\_\_}.
Sva imena koja počinju tako dobijaju kraće ime u procesu minifikacije.
