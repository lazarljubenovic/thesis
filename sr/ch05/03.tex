\section{Anatomija fabrika komponenti}\label{sec:anatomija-fabrika-komponenti}

Fabrika komponente je funkcija koja vraća objekat.
Nad objektom su definisane neke metode koje se pozivaju tokom životnog ciklusa komponente, kao i neka svojstva u koja se smešta stanje komponente koje se koristi prilikom izvršenja metoda.

Svaka fabrika ima najviše četiri glavne metode:

\begin{itemize}
\item \code{init}\footnote{Metoda \code{init} se u kodu generiše kao metoda \code{\_\_wane\_\_init}, kao i sve ostale metode i svojstva koja se pominju u ovom i drugim odeljcima. Prefiks \code{\_\_wane\_\_} koristi se da bi se omogućila minifikacija (v. \cref{subsec:bandlovanje}). Ovde se metode navode bez prefiksa radi lakšeg čitanja i pisanja.} -- poziva se jednom, tokom inicijalizacije komponente,
\item \code{diff} -- poziva se posle inicijalizacije i svaki put u okviru metode za ažuriranje pogleda,
\item \code{update} -- poziva se svaki put kada se primeti da je došlo do potencijalne promene u modelu, i
\item \code{destroy} -- poziva se jednom, kada komponentu treba uništiti i ukloniti iz DOM-a.
\end{itemize}

U zavinosti od rezultata analize, neke metode se mogu izbaciti i time se smanjuje kôd aplikacije.

\subsection{Metoda \code{init}}

Metoda \code{init} služi da kreira neophodne DOM čvorove i inicijalizuje instancu klase deklarisane od strane autora, kao i da prvim pozivom metode za traženje razlike \code{diff} omogući ažuriranje komponente.

\subsubsection{Koren komponente}
\label{subsubsec:koren-komponente}
\label{subsubsec:init-part-1}

Koren komponente je čvor u DOM-u za koji se komponenta vezuje -- koreni čvorovi pogleda vezuju se za koren kao njegova deca.
Smešta se u promenljivu \code{root}.

U slučaju da se radi o pristupnoj komponenti (\cref{subsec:pristupna-komponenta}), za koren se koristi \code{body} element.

\begin{lstlisting}
this.root = document.body
\end{lstlisting}

U svim ostalim slučajevima, koren se uzima iz svojstva \code{domNodes} roditeljske fabrike (v. \cref{subsubsec:registrovanje-dece}), navođenjem odgovarajućeg indeksa.
Indeks se dobija pribavljanjem sidra (v. \cref{sec:getSelfBindings}) i određivanjem njegovog indeksa (v. \cref{sec:getSingleIndexFor}).

\begin{lstlisting}
this.root = this.factoryParent.domNodes[0]
\end{lstlisting}

\subsubsection{Instanca klase}
\label{subsubsec:instanca-klase}
\label{subsubsec:init-part-2}

U instanci klase nalaze se metode koje vrše manipulaciju nad stanjem ili ispaljuju događaje na koje preci komponente u stablu mogu da se pretplate definisanjem poveza za njene izlaze.
Fabrici je potrebna ova instanca kako bi bilo moguće odrediti razliku u stanju tokom procesa ažuriranja u metodu \code{diff} (v. \cref{subsec:metoda-diff}), kako bi se mogle čitati vrednosti, i kako bi mogle da se pozivaju metode definisane od strane autora.

U najjednostavnijem slučaju se klasa jednostavno instancira korišćenjem operatora \code{new}.

\begin{lstlisting}
this.data = new Component()
\end{lstlisting}

Međutim, u slučaju da je potrebno ubrizgati fabriku u komponentu radi asinhronih ažuriranja (v. \cref{subsubsec:ubrizgavanje-fabrike}), konstruktor se može pozvati i sa argumentom \code{this}.

\begin{lstlisting}
this.data = new Component(this)
\end{lstlisting}

\subsubsection{Prihvatanje podataka kroz ulaze i izlaze}
\label{subsubsec:init-part-3}

U komponentu se mogu proslediti podaci pomoću ulaza i mogu se osluškivati događaji pomoću izlaza.

Kada se kreira instanca klase koja predstavlja komponentu, pribavljaju se ulazni podaci od fabrika-predaka.
Pribavljanje se obavlja jednostavnim operatorom dodele.

\begin{lstlisting}
// /@deo šablona@/
<Component [a]="b"/>

// generisan kod
this.data.a = this.factoryParent.b
\end{lstlisting}

Mada sintaksno izlazi izgledaju drugačije od ulaza, i mada se nameće drugačija semantika, izlazi se interno tretiraju vrlo slično ulazima.
Zapravo, jedini razlog zbog koga se ne treiraju apsolutno identično u slučaju da nema argumenata jeste ponašanje ključne reči \code{this} u Javaskriptu.
Zato se umesto puke dodele uvek pravi streličasta funkcija.

\begin{lstlisting}
// /@deo šablona@/
<Component (a)="b()"/>

// generisan kod
this.data.a = () => {
  this.factoryParent.b()
}
\end{lstlisting}

U slučaju da su pozivu funkcije u okviru poveza prosleđeni argumenti, i oni se obračunavaju kao i sva ostala svojstva i metode.

\begin{lstlisting}
// /@deo šablona@/
<Component (a)="b(c)"/>

// generisan kod
this.data.a = () => {
  this.factoryParent.b(this.factoryParent.c)
}
\end{lstlisting}

Ukoliko se kao argument nađe \code{\#}, on se prosleđuje kao argument funkcija koja se dodeljuje izlazu.

\begin{lstlisting}
// /@deo šablona@/
<Component (a)="b(#)"/>

// generisan kod
this.data.a = (placeholder) => {
  this.factoryParent.b(placeholder)
}
\end{lstlisting}

\subsubsection{Kreiranje DOM čvorova}
\label{subsubsec:init-part-4}

Svi DOM čvorovi koji su deklarisani šablonskom sintaksom u sklopu komponente instantiraju se i smeštaju u niz kako bi im se kasnije moglo pristupiti.

Za njihovo instanciranje koriste se pomoćne metode koje samo delegiraju zadatak HTML metodama \code{createElement}, \code{createTextNode} i \code{createComment} definisanim nad tipom \code{Document} umesto direktnog pristupa.
Ovime je omogućeno da se za njihova imena veže prefiks~\code{\_\_wane\_\_} i time obavi minifikacija ovih imena~(v.~\cref{subsec:bandlovanje}).

\begin{lstlisting}
this.domNodes = [
  /*0*/ createTextNode('No entered data.'),
  /*1*/ createElement('button'),
  /*2*/ createTextNode('Add some'),
];
\end{lstlisting}

\subsubsection{Sastavljanje DOM čvorova}
\label{subsubsec:init-part-5}

Sada kada elementi postoje, treba ih sastaviti tako da oponašaju autorovu definiciju kroz šablone.
Za ovo se takođe jednostavne pomoćne funkcije.
Stablo pogleda se obilazi rekurzivno.

\begin{lstlisting}
appendChildren(this.root, [
  this.domNodes[0],
  appendChildren(this.domNodes[1], [
    this.domNodes[2],
  ])
]);
\end{lstlisting}

\subsubsection{Registrovanje dece}
\label{subsubsec:registrovanje-dece}
\label{subsubsec:init-part-6}

Registrovanje dece se obavlja pomoćnom funkcijom \code{createFactoryChildren}.
Osim što se za tekuću fabriku vezuju prosleđena deca-fabrike, vrši se i povezivanje u obrnutom smeru -- deci se kao roditelj dodeljuje tekuća fabrika.

\begin{lstlisting}
createFactoryChildren(this, [
  ChildComponentFactory(),
])
\end{lstlisting}

Ukoliko fabrika nema decu i ukoliko fabrika nema \code{destroy} metodu (v. \cref{subsec:metoda-destroy}), ovaj deo koda se ne generiše.
Ukoliko metoda \code{destroy} postoji, neophodno je da postoji prazan niz \code{this.factoryChildren = []} kako bi rekurzivni obilazak prilikom uništavanja radio.

\subsubsection{Inicijalizacija svojstava i atributa u DOM-u}
\label{subsubsec:init-part-7}

Iako su HTML čvorovi kreirani, u opštem slučaju i dalje u potpunosti ne odgovaraju onome što je autor definisao jer nisu dodeljene vrednosti atributima i svojstvima u skladu sa deklaracijom šablona.

Koristeći indekse za pristup DOM čvorovima (v. \cref{sec:getSingleIndexFor}), štampaju se odgovarajuće dodele (za svojstva) i pozivi metoda (za atribute).
Ovo uključuje i obične deklaracije (\code{id="a"}) i poveze (\code{[id]="a"}).

\begin{lstlisting}
this.domNodes[3].type = 'text'
this.domNodes[3].className = 'field'
this.domNodes[3].setAttribute('aria-label', 'Name')
\end{lstlisting}

\subsubsection{Pretplata na događaje iz DOM-a}
\label{subsubsec:pretplata-na-dogadjaje-iz-doma}
\label{subsubsec:init-part-8}

Za pretplatu na DOM događaje koristi se pomoćna metoda koja delegira poziv standardnoj metodi \code{addEventListener}.
Osim poziva funkcije koju je autor definisao kroz šablon, dodaje se i niz poziva metod\=a \code{update} nad svim fabrikama nad kojim poziv autorove metode može uticati u skladu sa algoritmom definisanim u~\cref{fn:getFactoriesAffectedByCalling}.

\begin{lstlisting}
addEventListener(this.domNodes[1], 'submit', (placeholder) => {
  this.data.onSubmit(placeholder)
  this.factoryParent.factoryParent.factoryParent.update()
  this.factoryParent.update()
});
\end{lstlisting}

\subsubsection{Deklaracija metoda za asinhrono ažuriranje}
\label{subsubsec:init-part-9}

Slično kao kod pretplate, generišu se pozivi metod\=a \code{update}, s tim što se u ovom slučaju smeštaju u niz kako bi se određena metoda pozvala iz samog koda klase~(v.~\cref{subsubsec:ubrizgavanje-fabrike}).

\begin{lstlisting}
this.updateAsync = [
  /* 0 */ () => { this.update() },
  /* 1 */ () => { this.factoryParent.update() },
];
\end{lstlisting}

Ako asinhronog koda u komponenti nema, ovaj deo koda se u potpunosti izostavlja.

\subsubsection{Prva razlika}
\label{subsubsec:init-part-10}

Kako bi se pokrenuo proces računanja razlika između \textquote{prethodnog} i trenutnog stanja, mora se populisati \code{prevData} odgovorajućim vrednostima.
Najbrži način da se ovo uradi je prost poziv metode \code{diff} (v. \cref{subsec:metoda-diff}).

U specijalnim slučajevima kada komponenta nema stanje (ili su svi njegovi elementi proglašeni kao konstantni, v. \cref{sec:konstante}), ovaj deo koda se ne generiše jer tada metoda \code{diff} ne postoji.

\subsubsection{Primer}

Primer čitave metode \code{init} dat je u nastavku.

\begin{lstlisting}
init() {
  // /@\cref{subsubsec:init-part-1}: \nameref{subsubsec:init-part-1}@/
  this.root = this.factoryParent.domNodes[0]

  // /@\cref{subsubsec:init-part-2}: \nameref{subsubsec:init-part-2}@/
  this.data = new EmptyState()

  // /@\cref{subsubsec:init-part-3}: \nameref{subsubsec:init-part-3}@/
  this.data.createNew = () => {
    this.factoryParent.factoryParent.factoryParent
      .data.onCreateNewClick()
  };

  // /@\cref{subsubsec:init-part-4}: \nameref{subsubsec:init-part-4}@/
  this.domNodes = [
    /*0*/ createTextNode('\n  No entered data. '),
    /*1*/ createElement('button'),
    /*2*/ createTextNode('Add some!'),
  ]
  
  // /@\cref{subsubsec:init-part-5}: \nameref{subsubsec:init-part-5}@/
  appendChildren(this.root, [
    this.domNodes[0],
    appendChildren(this.domNodes[1], [
      this.domNodes[2],
    ]),
  ])

  // /@\cref{subsubsec:init-part-6}: \nameref{subsubsec:init-part-6}@/
  this.factoryChildren = [];

  // /@\cref{subsubsec:init-part-7}: \nameref{subsubsec:init-part-7}@/
  // Nema

  // /@\cref{subsubsec:init-part-8}: \nameref{subsubsec:init-part-8}@/
  addEventListener(this.domNodes[1], 'click', () => {
    this.data.createNew();
    this.factoryParent.factoryParent.factoryParent.update();
  })

  // /@\cref{subsubsec:init-part-9}: \nameref{subsubsec:init-part-9}@/
  // Nema

  // /@\cref{subsubsec:init-part-10}: \nameref{subsubsec:init-part-10}@/
  this.diff();
}
\end{lstlisting}

\subsection{Metoda \code{diff}}
\label{subsec:metoda-diff}

Metoda za traženje razlika jednostavno vraća objekat sa razlikama.
U pitanju je objekat tipa \code{Record<keyof C, boolean>}, gde je \code C klasa pridružena komponenti.
Drugim rečima, u pitanju je objekat kome su ključevi imena svojstava deklarisanih u klasi (odnosno neki njihov podskup), a vrednosti su \code{true} ili \code{false}.

Računanje razlike oslanja se na prethodnu vrednost stanja sačuvanu u \code{prevData} i poredi vrednosti sa trenutnom vrednošću sačuvanu u \code{data}.
Zahvaljujući jednom pokretanju funkcije \code{diff} \textquote{u prazno} (rezultat se odbacuje), ovaj pristup funkcioniše i prilikom prvom pokretanju ove metode.

Računanje razlike i ažuriranje prethodne vrednosti obavlja se istovremeno.

\begin{lstlisting}
value: this.prevData.value !== (this.prevData.value = this.data.value),
\end{lstlisting}

Najpre se izvršava leva strana poređenja (\code{this.prevData.value}), i ta vrednost se poredi sa rezultatom izvršenja operatora dodele (\code{=}) s desne strane (\code{this.prevData.value = this.data.value}).
Taj rezultat je vrednost upisana u \code{this.data.value}, pa se poređenje vrši između prethodne i trenutne vrednost.
Sporedni efekat primena operatora dodele je to što je sada trenutna vrednost upisana u \code{this.prevData.value}.
Kako se tokom jednog poziva funkcije \code{diff} ista vrednost neće koristiti dvaput, ovaj pristup funkcioniše.

Metoda se koristi u okviru metode \code{update} (v. \cref{subsec:metoda-update}).
Ne generiše se ukoliko nema stanja za koje ima potrebe računati razliku.

\subsection{Metoda \code{update}}
\label{subsec:metoda-update}

Pre nego što se započne sa bilo kakvim ažuiranjem, preračunava se razlika trenutnog stanja u odnosu na prethodno.
Ovo se obavlja pozivom metode \code{diff} (v. \cref{subsec:metoda-diff}).
Rezultat se čuva u lokalnoj konstanti \code{diff}.

Za generisanje koja koji vrši ažuriranje koriste se mape razlika (v. \cref{sec:getDomDiffMap}).

\subsubsection{Ažuriranje DOM elemenata}

Iz mape razlika čitaju se indeksi DOM čvorova za koje su vezani povezi.
Zatim se vrši uslovno ažuriranje na osnovu imena svojstva iz klase i rezultata traženje razlika.

\begin{lstlisting}
if (diff.title) {
  this.domNodes[2].data = this.data.title;
}
if (diff.value) {
  this.domNodes[3].value = this.data.value;
}
\end{lstlisting}

\subsubsection{Ažuriranje potomaka}

Slično kao kod DOM elemenata, za fabrike se takođe dobavlja indeks i skup poveza.
Nakon obavljenog ažuriranja, poziva se metoda \code{update} nad fabrikom čije je stanje ažurirano.

Da bi se obezbedilo da se ažuriranje deteta obavlja samo jedno, vrši se grupisanje uslova operatorom \code{||}.

\begin{lstlisting}
if (diff.visibleItems || diff.items) {
  this.factoryChildren[0].data.items = this.data.visibleItems;
  this.factoryChildren[0].data.totalCount = this.data.items.length;
  this.factoryChildren[0].update();
}
\end{lstlisting}

\subsection{Metoda \code{destroy}}
\label{subsec:metoda-destroy}

\code{destroy} je jednostavna metoda koja najpre poziva ovu metodu za decu (kako bi se proces uništenja dogodio rekurzivno, počev od fabrika koje su listovi), a zatim prolazi kroz sve DOM elemente koji su direktni potomci korena i uklanja ih.

\begin{lstlisting}
destroy() {
  this.factoryChildren.forEach(factoryChild => {
    factoryChild.destroy()
  });
  while (this.root.firstChild !== null) {
    this.root.removeChild(this.root.firstChild)
  }
}
\end{lstlisting}
