\section{Fabrike uslovne direktive (\code{w:if})}\label{sec:fabrike-uslovne-direktive}

Fabrike uslovne direktive sastoje se iz dva dela.
Jedan deo se stara za to da se sadržaj unutar direktive stvori i uništi u trenutku kada se primeni vrednost vezana za \code{w:if}, a drugi deo se stara da sadržaj koji se stvori bude onaj sadržaj koji je autor definisao.

Zajednički deo se naziva \textbf{uslovni pogled} (\textsl{conditional view}), a deo specifičan za svaku instancu direktive naziva se \textbf{parcijalni pogled} (\textsl{partil view}).

\subsection{Uslovni pogled}

Komponente imaju koreni čvor unutar koga se kreira njen sadržaj.
Ovaj čvor slući da nedvosmisleno definiše početak i kraj komponente.
Kada komponentu treba uništiti, jasno je određeno koji čvorovi treba da budu oklonjeni iz DOM-a.

U slučaju direktiva, proizvoljan skup DOM čvorova može biti obuhvaćen direktivom.
Ne samo da se može javiti više HTML elemenata, već neki čvorovi uopšte ne moraju da budu elementi -- mogu biti tekstualni čvorovi.

Kako bi se znalo odakle počinje i gde se završava sadržaj direktive, u DOM se ubacuju dva \textbf{oluka} (\textsl{outlet}).
Oluk služi da usmeri delove koda na to gde se tačno određeni deo pogleda (parcijalni pogled) nalazi.
Prvi oluk (po redosledu dokumenta) naziva se \textbf{početni} a drugi \textbf{krajnji}.

Četiri metode se definišu za uslovne poglede: metoda za inicijalizaciju, metoda za inicijalizaciju parcijalnog pogleda, metoda za ažuriranje i metoda za uništenje.

\subsubsection{Metoda \code{init}}

Prilikom inicijalizacije uslovnog pogleda, slično komponentama, najpre se dobavlja koren -- samo je u ovom slučaju to oluk koga čine dva komentara.
Umesto dobavljanja jednog DOM čvora iz roditelja, dobavljaju se dva -- jedan je početni komentar, a drugi krajnji.

\begin{lstlisting}
this.openingCommentOutlet = this.factoryParent.domNodes[4]
this.closingCommentOutlet = this.factoryParent.domNodes[5]
\end{lstlisting}

Za razliku od komponenti, gde se \code{data} fabrike određuje na osnovu autorovog koda (v. \cref{subsubsec:instanca-klase}), kod direktiva se unapred zna kog je oblika stanje.
U slučaju \code{w:if}, to je jedna logička vrednost.
Ona se čuva u promenljivu \code{data}.
Prethodno stanje čuva se u promenljivoj \code{prevData}, isto kao i kod fabrika komponenti.
U metodi \code{init} se ova dva svojstva inicijalizuju na vrednost koja im je prosleđena.

\begin{lstlisting}
this.data = this.prevData = !this.factoryParent.data.isEmpty
\end{lstlisting}

Zatim se, u slučaju da je vrednost \code{this.data} istinita, odmah poziva metoda za inicijalizaciju parcijalnog pogleda.
Inače se proces inicijalizacije okončava.
Sadržaj uslovne direktive pojaviće se kada se \code{this.data} promeni u istinitu vrednost pokretanjem metode \code{update}.

\begin{lstlisting}
if (this.data) {
  this.initPartialView()
}
\end{lstlisting}

\subsubsection{Metoda \code{initPartialView}}

Za incijalizaciju samog pogleda unutar direktive, tj. parcijalnog pogleda koji je vezan za direktivu, koristi se metoda \code{initPartialView}.
Kao prvo i jedino dete tekuće fabrike inicijalizuje se parcijalni pogled i prosleđuju mu se oluci.
Zatim se inicijalizuje sâm parcijalni pogled pozivom njegove metode \code{init}.

\begin{lstlisting}
this.factoryChildren[0] = PartialView_ConditionalView_isEmpty_5_4()
this.factoryChildren[0].factoryParent = this
this.factoryChildren[0].openingCommentOutlet = this.openingCommentOutlet
this.factoryChildren[0].closingCommentOutlet = this.closingCommentOutlet
this.factoryChildren[0].init()
\end{lstlisting}

\subsubsection{Metoda \code{update}}

Ažuriranje uslovnog pogleda svodi se na ispitivanje koja od četiri moguće kombinacije vrednosti svojstava \code{prevData} i \code{data} je u pitanju.

Kada su obe vrednosti \code{false}, znači da se direktiva nalazila u \textquote{isključenom} stanju i da ostaje u njemu.
U ovoj situaciji ne treba uraditi ništa.

Prelazak iz \code{false} u \code{true} znači da je direktiva bila \textquote{isključena} i da se ovim ažuriranjem ona \textquote{uključuje}.
Ovaj slučaj se obrađuje tako što se pokrene metoda \code{initPartialView} kojom se sadržaj direktive kreira i dodaje u DOM.

Prelazak iz \code{true} u \code{false} znači da je direktiva bila \textquote{uključena} i da se ovim ažuriranjem ona \textquote{isključuje}.
Tada se parcijalni pogled treba uništiti, pa se poziva ova metoda.

Obe vrednosti \code{true} znači da samo treba ažurirati parcijalni pogled.

\begin{lstlisting}
update(diff) {
  const prev = this.prevData;
  this.prevData = this.data;
  if (!prev && !this.data)
    return;
  if (!prev && this.data)
    return this.initPartialView();
  if (prev && !this.data)
    return this.factoryChildren[0].destroy();
  this.factoryChildren[0].update(diff);
}
\end{lstlisting}

\subsubsection{Metoda \code{destroy}}

Metoda za uništavanje delegira zadatak parcijalnom pogledu, ukoliko on postoji u trenutku kada se ova metoda pozove.

\begin{lstlisting}
destroy() {
  if (this.factoryChildren[0] != null) {
    this.factoryChildren[0].destroy();
  }
}
\end{lstlisting}

\subsection{Parcijalni pogled}

Parcijalni pogled po mnogo čemu podseća na komponentu.
Može se reći i da je komponenta je specijalni slučaj parcijalnog pogleda.
Glavna razlika je u tome što ne postoji \code{data} svojstvo.
Svi podaci koji se mogu iskoristiti u parcijalnom pogledu dolaze iz komponenti i direktiva (i to ne \code{w:if}) koje se nalaze iznad tekućeg čvora u stablu fabrika.

Pored toga, razlikuje se način vezivanja kreiranih DOM čvorova za postojeći DOM, ali samo u prvom nivou.
Pošto koren nije jedan element, umesto \code{appendChild} koristi se \code{insertBefore} nad krajnjim olukom.
U skladu sa ovime razlikuje se i \code{destroy} metoda.

Primer celog parcijalnog pogleda dat je u nastavku.

\begin{lstlisting}
var PartialView_ConditionalView_isEmpty_5_4 = () => ({
  prevData: {},
  init() {
    this.domNodes = [
        /*0*/ createElement('button'),
        /*1*/ createTextNode('Save'),
        // ...
    ]
    insertBefore(this.closingCommentOutlet, [
        appendChildren(this.domNodes[0], [
            this.domNodes[1],
            // ...
        ])
    ])
    this.factoryChildren = []
    this.domNodes[0].type = 'submit'
  },
  diff() {
      return {
        // ...
      }
  },
  update(diff) {
      Object.assign(diff, this.diff())
      // ...
  },
  destroy() {
      this.factoryChildren.forEach(child => child.destroy())
      let node = this.openingCommentOutlet.nextSibling
      while (node != this.closingCommentOutlet) {
          const nextNode = node.nextSibling
          node.remove()
          node = nextNode
      }
  },
})
\end{lstlisting}

 