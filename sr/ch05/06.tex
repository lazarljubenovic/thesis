\section{Fabrike direktive za ponavljanje (\code{w:for})}
\label{sec:fabrike-direktive-za-ponavljanje}

Konceptualno, direktiva za ponavljanje funkcioniše isto kao i uslovna direktiva -- sastoji se iz dva dela, pri čemu se unutrašnji deo takođe zove \textbf{parcijalni poged} i ima identičnu strukturu kao i parcijalni pogled vezan za uslovnu direktivu, a spoljni deo \textbf{ponavljajući pogled} (\textsl{repeating view}).

Struktura ponavljajućeg pogleda je, međutim, znatno drugačija od strukture uslovnog pogleda.
Sadrži najdinamičniji deo koda iz celog frejmvorka; zbog prirode direktive nemoguće mnogo stvari odrediti statički jer je nemoguće predvideti kako će se dodavati elementi u niz, brisati iz njega, i kako će menjati mesta.
Ovo je verovatno najneoptimalniji deo generisanog koda, pa se u poređenju sa ostatakom može učiniti kao previše glomazan, ali se zapravo bazira na konceptima traženje razlike između dva niza i sličan je algoritmima za \textit{dom diff} koji koriste mnoge biblioteke.
Konkretno je inspirisan funkcijom za traženje i \textquote{krpljenje} razlika iz jedne od ranijih verzija mikro-frejmvorka \textit{Hyperapp}~\cite{hyperapp:patch}.

Algoritam funkcioniše tako što prolazi paralelno kroz stari i novi niz ključeva.
Prednost se daje starom nizu -- on se obilazi sve dok se ne naiđe na trenutni element iz novog niza, ili dok se ne obiđe ceo.
Svaki put kada se detektuje promašaj, element se čuva u objektu kako bi se kasnije pronašao jednostavnije ($O(1)$ umesto $O(n)$).

Ukoliko se element ne pronađe u celom starom nizu, to znači da je on novi u novom nizu, pa se odgovarajući DOM čvorovi kreiraju.
Mesto kreiranja određuje se jer se, osim kroz stari i novi niz, prolazi kroz indekse koji predstavljaju DOM čvorove, i to samo kada dođe do pogotka, zamene ili dodavanja novog elementa.

Ukoliko se element pronađe, bilo direktno iz niza ili kao stara keširana vrednost, on se, u zavinosti od njegove pozicije u DOM-u, premešta ili ostavlja na isto mesto.
U oba slučaja se vrši ažuriranje pogleda tog elementa.

Na ovaj način se, u praksi, dobar deo čvorova koristi ponovo.
Algoritam je daleko od optimalnog iz ugla najmanjeg broja operacija koje je neophodno izvršiti.
Učinjen je kompomis po pitanju brzine izvršenja algoritma i operacija, jer je neophodno da se promene u DOM-u dešavaju u realnom vremenu.

% malo je nezgodno da se celo objasni onako striktno ali uz jednostavne primere koji ilustruju par ključnih slučejeva, neki edge case i jedan-dva kompletna primera koji ilustruju ceo obilazak i rezonovanje, i naravno pre svega toga idejno rešenje bi trebao da bude dovoljno jasno
