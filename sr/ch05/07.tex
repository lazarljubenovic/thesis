\section{Stilovi}\label{sec:ch05:stilovi}

Kao što je već pomenuto u \cref{sec:ch04:stilovi}, stilovi se enkapsuliraju, tj. vezani su samo za komponentu kojoj su pridruženi.
Ovo je implementirano tako što se svim HTML čvorovima dodeljuje isti atribut, jedinstven za svaku komponentu, a CSS se transformiše tako da se uz selektore doda taj atribut.

\begin{lstlisting}
// autorov kod
@Template(`
  <span className="one">
    Some <strong>text</strong> here
  </span>
  <span className="two">
    And <strong>more</strong>
  </span>
`)
@Style(`
  span { color: red }
  .one { text-style: italic }
  span.two > strong { text-decoration: underline }
`)
class Example { }

// /@DOM tokom izvršenja@/
<w-example>
  <span className="one" data-w-1>
    Some <strong data-w-1>text</strong> here
  </span>
  <span className="two" data-w-1>
    And <strong data-w-1>more</strong>
  </span>
</w-example>

// /@Stilovi tokom izvršenja@/
span[data-w-1] { color: red }
.one[data-w-1] { text-style: italic }
span.two[data-w-1] > strong[data-w-1] {
  text-decoration: underline
}
\end{lstlisting}

Osim ovoga, transformiše se i ime komponente.

\begin{lstlisting}
// autorov kod
@Register(Child)
@Template(`
  <span>Text</span>
  <Child/>
`)
@Style(`
  span  { background-color: red }
  Child { background-color: blue }
`)
class Example { }

// /@DOM tokom izvršenja@/
<w-example>
  <span data-w-1>Text</span>
  <w-child data-w-1><!-- ... --></w-child>
</w-example>

// /@Stilovi tokom izvršenja@/
span[data-w-1] { background-color: red }
w-child[data-w-1] { background-color: blue }
\end{lstlisting}

Da bi se ovo obezbedilo, tokom inicijalizacije svojstava i atributa u DOM-u (v. \cref{subsubsec:init-part-7}) se dodaju i atributi oblika \code{data-w-N}, gde je \code N jedinstven broj za svaku komponentu.

\begin{lstlisting}
this.domNodes[0].setAttribute('data-w-1', '')
this.domNodes[2].setAttribute('data-w-1', '')
this.domNodes[3].setAttribute('data-w-1', '')
\end{lstlisting}
