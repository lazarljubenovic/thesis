\chapter{Kompilacija Vejn aplikacije}

Jedinstvenost Vejna ogleda se u tome što kôd frejmvorka, u tradicionalnom smislu -- ne postoji.
Mada iz ugla developera izgleda da koristi kôd frejmvorka tokom izvršenja, preciznije je reći da je Vejn \textbf{kompilator}.

Kod napisan od strane developera se koristi da bi se statičkom analizom utvrdile zavisnosti koje postoje između modela i šablona komponente, kao i između komponenti međusobno.
Na osnovu ovoga se generiše kod aplikacije koji se koristi tokom izvršenja.

Gledano na najvišem nivou, kompilator čine dva dela: \textbf{analizator} izvornog koda i \textbf{generator} rezultujućeg koda.

\subfile{ch05/01}
\subfile{ch05/02}
\subfile{ch05/03}
\subfile{ch05/04}
\subfile{ch05/05}
\subfile{ch05/06}
\subfile{ch05/07}
\subfile{ch05/08}
