\chapter{Konkurencija i poređenje}

Ideja o transformaciji i generisanju koda u kontekstu front-end developmenta svakako nije nova -- zapravo, može se reći da ono što održava front-end development jeste upravo bild proces.

\section{Jezici koji se kompajliraju u Javaskript}

Jezici koji se kompajliraju u Javaskript postaju popularni zajedno sa pojavom SPA kao obrasca za izradu aplikacija.
Prvi široko rasprostranjen takav jezik je \textbf{Kofiskript} (\textsl{CoffeeScript}). % https://coffeescript.org/
Nastao je 2009, a najveću popularnost ima od 2011 do 2014, kada polako počinje da pada u zaborav.
U SOJS anketi iz 2016, 60\% ljudi je reklo da je \textquote{čulo za njega, ali ne želi da ga uči}, dok 25\% kaže da su \textquote{ga koristili ranije, ali više ne} -- samo 6\% je bilo zadovoljno jezikom.

Pad Kofiskripta može da se protumači kao pozitivna stvar, kada se uzme u obzir vreme i razlog zbog koga je on nastao.
Naime, razlog zbog koga je postao popularan bile su one strukture koje su kasnije došle uz Ekmaskript 2015.
Razlika je u tome što Kofiskript nije \textit{standard}, niti se oslanja na njega -- radi se o potpuno proizvoljnoj sintaksi, inspirisanoj Rubijem, Pajtonom i Haskelom.
Uz dolazak novih standarda, novih alata i novih jezika, Kofiskript nema šta da pruži.
Bez načina za definisanje tipova podataka, nema prednost u ranijem otklanjanju grešaka iz koda, a bez novih osobina u odnosu na novi ES standard, nema razloga za korišćenje.
Konciznija i kraća sintaksa, umesto prednosti, počinje da bude dodatni korak koji developeri moraju da savladaju pre nego što mogu da počnu sa radom. % https://github.com/jashkenas/coffeescript/issues/4288

Ipak, Kofiskript igra značajnu ulogu u razviću samog Javaskripta.
Daglas Krokford izjavio je 2012. da je Kofiskript \textquote{lep, elegantan i minimalan} i da bi \textquote{voleo da je Javaskript više nalik Kofiskriptu}, nazivajući ga \textquote{ljupkim malim jezikom}. % https://jaxenter.com/douglas-crockford-java-was-a-colossal-failure-javascript-is-succeeding-because-it-works-105395.html
Naime, Krokford je bio šampion predloga za \textquote{debele streličaste funkcije} u Javaskriptu (\code{=>}), a deo ideje za konciznu sintaksu je dobio upravo od Kofiskripta. % https://news.ycombinator.com/item?id=9266517 

Još jedan popularan jezik koji se pojavljuje je \textbf{Dart}; u pitanju je Guglov proizvod iz 2011. godine.
Kasnije dobija zvaničnu Ekma specifikaciju i poseban komitet, TC52.
Slično Kofiskriptu, ima proizvoljnu sintaksu -- ovog puta, inspirisanu C-olikim jezicima.
Umesto kraćeg zapisa Javaskript konstrukcija, Dart se osvrće na prvoklasnu podršku za obrasce kao što su klase, miksini, apstraktne klase, itd.
Pored ovoga, uvodi i sistem tipova -- sistem koji naziva \textquote{sigurnim}: koristi \textquote{kombinaciju statičke provere tipova i provere tokom izvršenja kako bi se postarao da vrednost koja je dodeljena promenljivoj uvek odgovara statičkom tipu promenljive}. % https://www.dartlang.org/guides/language/sound-dart

Istovremeno, Majkrosoft radi na projektu pod kodnim imenom \textquote{Strada}.
Među timom od oko 50 ljudi koji su radili na projektu, ističe se ime Anders Hajlzberg -- autor Turbo Paskala i Delfija, i glavni arhitekta u timu za C$\sharp$ u Majkrosoftu.
Jezik je prvog oktobra 2012. godine objavljen pod nazivom \textbf{TajpSkript}, u verziji 0.8.
Prva stabilna verzija (1.0) objavljena je 2014. godine.
