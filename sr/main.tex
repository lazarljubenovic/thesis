\documentclass[twoside, openright]{report}
\usepackage{hyperref}
\usepackage[utf8]{inputenc}
\usepackage[english,serbian]{babel}
\usepackage{parskip}
\usepackage{microtype}
\usepackage{subfiles}
\usepackage{csquotes}
\usepackage{xcolor}
\usepackage{epigraph}
\usepackage{graphicx}
\usepackage{bchart}

\usepackage{listings}
\usepackage{tikz}
\usetikzlibrary{calc, shapes, arrows, arrows.meta, positioning, fit}

% https://tex.stackexchange.com/questions/172415/upside-down-syntax-trees-for-linguistics-with-horizontal-lines
% \usepackage{forest}
% % Node shape adapted from http://www.texample.net/tikz/examples/data-flow-diagram/
% \makeatletter \pgfdeclareshape{myunderline}{
%   \inheritsavedanchors[from=rectangle]
%   \inheritanchorborder[from=rectangle]
%   \foreach \from in
%     {center,base,north,north east,east,south east,south,south west,west,north west}{
%       \inheritanchor[from=rectangle]{\from}
%   }
%   \backgroundpath{
%     \southwest \pgf@xa=\pgf@x \pgf@ya=\pgf@y
%     \northeast \pgf@xb=\pgf@x \pgf@yb=\pgf@y
%     % This can be improved by removing magic numbers
%     \pgfpathmoveto{\pgfpoint{\pgf@xa}{\pgf@ya+1.75em}}
%     \pgfpathlineto{\pgfpoint{\pgf@xb}{\pgf@ya+1.75em}}
%  }
% } \makeatother


% front, main, back matter za report
% https://tex.stackexchange.com/questions/154646/is-there-an-easy-way-to-get-the-frontmatter-mainmatter-and-backmatter-in-a-l
\makeatletter

\newcommand\frontmatter{%
    \cleardoublepage
  %\@mainmatterfalse
  \pagenumbering{roman}}

\newcommand\mainmatter{%
    \cleardoublepage
 % \@mainmattertrue
  \pagenumbering{arabic}}

\newcommand\backmatter{%
  \if@openright
    \cleardoublepage
  \else
    \clearpage
  \fi
 % \@mainmatterfalse
   }

\makeatother

% Watermark za draft verziju
% \usepackage[printwatermark]{xwatermark}
% \newsavebox\mybox
% \savebox\mybox{\tikz[color=black,opacity=0.2,font=\sffamily\bfseries]\node{DRAFT};}
% \newwatermark*[
%   allpages,
%   angle=45,
%   scale=10,
%   xpos=-32,
%   ypos=15
% ]{\usebox\mybox}

% Make "subsubsection" have numbers
\setcounter{secnumdepth}{4}

% Bibliography
\usepackage{biblatex}
\addbibresource{mdn.bib}
\addbibresource{w3c.bib}
\addbibresource{other.bib}

% Some listings settings
\lstset{
  basicstyle=\ttfamily\small,
  columns=fullflexible,
  escapeinside={/@}{@/}
}

% For drawing arrows and boxes over code
\newcommand{\tikzmark}[1]{\tikz[overlay, remember picture] \node (#1) {};}
\newcommand{\WrapInBox}[2]{\fill [black, opacity=0.05] ([shift={(0, -6pt)}]#1.north west) rectangle ([shift={(0, 6pt)}]#2.north east);}

% Definition of arrow drawings in codes
\newcommand\DrawBox[2]{%
\begin{tikzpicture}[remember picture,overlay]
\draw[red] 
  ([yshift=1.7ex,xshift=-2pt]pic cs:#1) rectangle ([yshift=-.25ex,xshift=2pt]pic cs:#2);
\end{tikzpicture}%
}

% Pretty tables stuff
% https://tex.stackexchange.com/questions/163061/help-with-a-booktabs-table
\usepackage{booktabs,tabularx}

% For drawing the syntax trees
% https://tex.stackexchange.com/questions/5073/making-a-simple-directory-tree
\usepackage{forest}

% Fancy references to other sections within the doc
% https://tex.stackexchange.com/questions/208933/how-to-show-symbol-when-i-refer-a-chapter
\usepackage{cleveref}
\crefformat{section}{\S#2#1#3}
\crefformat{subsection}{\S#2#1#3}
\crefformat{subsubsection}{\S#2#1#3}

% Custom font
% https://tex.stackexchange.com/questions/5443/use-another-monospaced-font
\usepackage[T1]{fontenc}
\usepackage[scaled]{beramono}
% https://www.sharelatex.com/learn/Font_typefaces
\usepackage{charter}

% Remove "Chapter N" from chapter titles
% https://tex.stackexchange.com/questions/120740/how-do-i-remove-chapter-n-from-the-chapter-titles-of-a-book
\usepackage{titlesec}
\titleformat{\chapter}[display]
  {\normalfont\bfseries}{}{0pt}{\Huge}

% For the title page
\newcolumntype{R}{>{\raggedleft\arraybackslash}X}

%% Command \code
\newcommand{\code}[1]{\textcolor{black!80}{\texttt{#1}}}

% Tikz definitions for a flowchart (FlowChart => fc)
\tikzstyle{fc:block} = [
  rectangle,
  draw,
  fill=black!5,
  text centered,
  rounded corners,
  minimum height=2em,
  inner sep=1em
]
\tikzstyle{fc:line} = [draw, -latex]

% default text in tikz text
\tikzset{
  font=\small\sffamily
}

\title{Primena statičke analize za generisanje koda jednostraničnih veb-aplikacija}
\author{Lazar Ljubenović}

\DeclareQuoteStyle{serbian}% I looked it up on Wikipedia, no idea if it's right
  {\quotedblbase}
  {\textquotedblright}
  [0.05em]
  {\textquoteleft}
  {\textquoteright}


\begin{document}

\pagenumbering{gobble}
   
\begin{titlepage}
  \begin{center}

    % \includegraphics[width=2.6cm]{univerzitet_logo}%
    \begin{minipage}[b]{0.7\textwidth}
        \centering
        \textbf{UNIVERZITET U NIŠU\\ELEKTRONSKI FAKULTET}
    \end{minipage}%
    % \includegraphics[width=2.6cm]{fakultet_logo}
    
    \vspace{5cm}
    \LARGE
    \textbf{Primena statičke analize za generisanje koda jednostraničnih veb-aplikacija}
    
    \vspace{1cm}
    \normalsize
    \textls{MASTER RAD}
    \\
    \vspace{.33cm}
    \textit{Studijski program:} Računarstvo i informatika
  \end{center}

  % \vspace{1cm}
  \vfill

  \noindent 
  \begin{tabularx}{\textwidth}{XR}
    \textbf{Kandidat}                    & \textbf{Mentor}                \\
    dipl. inž. Lazar \textsc{Ljubenović} & doc. dr Ivan \textsc{Petković}
  \end{tabularx}

  \vfill

  \begin{center}
    Niš, 2018.
  \end{center}
\end{titlepage}

%\shipout\null

\frontmatter

\pagenumbering{gobble}

Master rad\\
PRIMENA STATIČKE ANALIZE ZA GENERISANJE\\KODA JEDNOSTRANIČNIH VEB-APLIKACIJA
\\

\vspace{0.5cm}


\textbf{Zadatak:} \textit{Upoznati se sa stanjem ekosistema Javaskript frejmvork\=a i biblioteka koje se koriste za izgradnju jednostraničnih veb-aplikacija, kao i alat\=a i jezik\=a koji se koriste tokom njihovog razvitka.
Istražiti kako se tehnike statičke analize mogu primeniti u njihovom razviću u cilju generisanja minimalne veličine koda.
Implementirati prototip rešenja, opisati ga, i na primeru aplikacije pokazati rezultate.
Uporediti dobijene rezultate sa rezultatima koje daju frejmvorci prihvaćeni u industriji i izvršiti analizu uočenih razlika.
Dati uvid u to kako se prototip može dalje razvijati i opisati njegov potencijal.}

\vfill

\textbf{Student:} Lazar Ljubenović, br. ind. 688

\vspace{0.5cm}

\textbf{Komisija za odbranu:}

\vspace{1cm}

\begin{tabular}{@{}ll}  
  \begin{tabular}{@{}ll}
    %1. & \underline{\hspace{5cm}}\\
    %2. & \underline{\hspace{5cm}}\\
    %3. & \underline{\hspace{5cm}}
    1. \underline{\hspace{6cm}} \\
    \\
    \\
    2. \underline{\hspace{6cm}} \\
    \\
    \\
    3. \underline{\hspace{6cm}}
  \end{tabular}  
    
&

  \begin{tabular}{ll}
    Datum prijave: & \underline{\hspace{3cm}}\\
    \\
    \\
    Datum predaje: & \underline{\hspace{3cm}}\\
    \\
    \\
    Datum odbrane: & \underline{\hspace{3cm}}
  \end{tabular}

\end{tabular}

\newpage

\begin{abstract}
  \subfile{ch00/abstract.sr}
\end{abstract}

\begin{otherlanguage}{english}
  \begin{abstract}
    \subfile{ch00/abstract.en}
  \end{abstract}
\end{otherlanguage}

\tableofcontents

\mainmatter

\subfile{ch01/main}

\subfile{ch02/main}

\subfile{ch03/main}

\subfile{ch04/main}

\subfile{ch05/main}

\subfile{ch06/main}

\subfile{ch07/main}

\appendix

\printbibliography[title=Literatura]

\end{document}
